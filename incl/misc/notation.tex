\chapter*{Notation and Shorthands}
\textbf{General Notation}
\begin{table}[H]
	\begin{tabular}{ll}
		$\chi, \gamma, \psi$ & Finite colorings on some set $A$.         \\
		$[1; n]$       & The set $\left\{1, 2, \ldots, n\right\}$. \\
	\end{tabular}
\end{table}
\textbf{Graph Ramsey Theory}
\begin{table}[H]
	\begin{tabular}{ll}
		$V(G), E(G)$                               & \makecell{The sets of vertices and edges of a graph $G$.}                                                  \\
		$K_{n}, K_n^{*}$                           & \makecell{A complete graphs on $V(K_n^{*}) = [0; n - 1]$ and $V(K_n) = [1; n]$ respectively.}              \\
		$G \vert_U$                                & \makecell{The graph $(U, E(G) \cap (U \times U))$ where $U \subseteq V(G)$.}                               \\
		$\mathcal{C}_{\chi}(G; \ell)$              & \makecell{The set of monochromatic cliques in $G$  under $\chi$ of order $\ell$.}                          \\
		$\mathcal{N}_{\chi}(v; c)$                 & \makecell{The set of neighbours of $v$, adjacent through a $c$ colored edge.}                              \\
		$n \to (\ell_1, \ell_2, \ldots, \ell_{r})$ & \makecell{Means there exists an $i$ such that every $r$-edge coloring on $K_n$                             \\ admits a clique of an appropriate size} \\
		$R(\ell_1, \ell_2, \ldots, \ell_{r})$      & The non-generalized Ramsey number associated with $\ell_1, \ell_2, \ldots, \ell_r$.                        \\
		$R(\ell; r)$                               & The Ramsey number $R(\underset{r \text{ times }}{\underbrace{\ell, \ell, \ldots, \ell}})$.                 \\
		$R(G_1, G_2, \ldots, G_{r})$               & The generalized Ramsey number associated with $G_1, G_2, \ldots, G_r$.                                     \\
		$(\mathcal{P}, \mathcal{L}, I)$            & A point line geometry.                                                                                     \\
		$PG(2, q)$                                 & The projective plane of order $q$.                                                                         \\
		$G_q$                                      & The incidence graph of $PG(2, q)$.                                                                         \\
		$H_q^{\preccurlyeq}$                       & \makecell{A graph with vertex set $E(G_q)$ defined using the total ordering $\preccurlyeq$ on $E(G_{q})$.}
	\end{tabular}
\end{table}

\textbf{Partition Regularity}
\begin{table}[H]
	\begin{tabular}{ll}
		$\mathcal{C}$       & A configuration over $\mathbb{N}^{+}$.                                                 \\
		$\mathcal{F}$       & A family of configurations over $\mathbb{N}^{+}$.                                      \\
	    $[m]_d$ & The remainder of $m$ when divided by $d$.  \\
		$AP_D$              & The family of arithmetic progressions with gaps in $D \subseteq \mathbb{N}^{+}$.       \\
		$W(k, r)$           & The van der Waerden number associated with $k, r \in \mathbb{N}^{+}$.                  \\
		$W^{*}(AP_D, k, r)$ & The strengthened van der Waerden number.                                                \\
		$AP_{(m, k)}$       & The family of $k$-term $(mod m)$-arithmetic progressions.                              \\
		$\chi_{\gamma, m}$  & The $r^m$-coloring of $[1; n]$ derived from the $r$ coloring $\gamma$ on $[1; n + m]$. \\
	  $S(k; r)$ & The Schur number associated with $k, r \in \mathbb{N}^{+}$ \\
	\end{tabular}
\end{table}
