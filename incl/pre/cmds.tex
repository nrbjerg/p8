% cmds.tex : custom commands and environments
% ------------------------------------------------------------------------------
% This file contains definitions for custom commands and environments, used for
% shorthand notation for macros used often in the project

% Mathematical symbols ---------------------------------------------------------
\newcommand{\N}{\mathbb{N}}          % natural numbers
\newcommand{\A}{\mathbb{A}}
\renewcommand{\P}{\mathbb{P}}
\newcommand{\Z}{\mathbb{Z}}          % integers
\newcommand{\Q}{\mathbb{Q}}          % rational numbers
\newcommand{\R}{\mathbb{R}}          % real numbers
\newcommand{\C}{\pazocal{C}}          % complex numbers
\newcommand{\K}{\mathbbm{k}}
\newcommand{\F}{\mathbb{F}}
\newcommand{\cF}{\overline{\F}}
\newcommand{\pols}{\K[X_1, X_2, \ldots, X_{n}]}
\newcommand{\fPols}{\F[\underline{x}_n]}
\newcommand{\Zar}{\mathcal{T}_Z}
\newcommand{\cZar}{\mathcal{F}_Z}
\renewcommand{\d}{\mathrm{d}}
\newcommand{\wt}{\mathrm{wt}}
\newcommand{\comp}{\mathbb{C}}
\newcommand{\ind}{\mathbbm{1}}       % indicator function
\newcommand{\bigO}{\mathcal{O}}      % big O
\newcommand{\ev}{\textnormal{Ev}_\mathcal{P}}
\newcommand{\Res}{\textnormal{Res}}
\newcommand{\Char}{\textnormal{char}}
\newcommand{\Rad}{\textnormal{Rad}}
\newcommand{\support}{\textnormal{supp}}
\newcommand{\ord}{\textnormal{ord}}
\newcommand{\diag}{\textnormal{diag}}

\newcommand{\bs}[1]{\boldsymbol{#1}}
\newcommand{\norm}[1]{\left\lVert#1\right\rVert}
\newcommand{\abs}[1]{\left\lvert#1\right\rvert}
\newcommand{\gen}[1]{\langle#1\rangle}
\newcommand{\floor}[1]{\left\lfloor#1\right\rfloor}
\newcommand{\ceil}[1]{\left\lceil#1\right\rceil}
\renewcommand{\bar}{\overline}
\renewcommand{\vec}[1]{\bm{#1}}      % bold vector style

\newcommand{\rank}{\textnormal{rank}}
\newcommand{\Null}{\textnormal{null}}
\newcommand{\row}{\textnormal{row}}
\newcommand{\Span}{\textnormal{span}}
\newcommand{\col}{\textnormal{col}}
\newcommand{\image}{\textnormal{image}}
% NOTE: \ker is already defined by latex.

% Theorem enviroments ----------------------------------------------------------
% http://www.ctex.org/documents/packages/math/amsthdoc.pdf

%BOKS:
% for adjustwidth environment
\usepackage[strict]{changepage}

% now constructions of the frames for theorems and such
\usepackage{framed}

\newenvironment{formal}{%
  \def\FrameCommand{%
    \hspace{4pt}%
    {\color{Black}\vrule width 2pt}%
    {\color{White}\vrule width 4pt}%
    %\colorbox{AAUblue2}%
  }%
   \MakeFramed{\advance\hsize-\width\FrameRestore}%
   \noindent\hspace{-4.55pt}% disable indenting first paragraph
   \begin{adjustwidth}{}{7pt}%
   \vspace{-18pt}%
}
{%
  \vspace{0pt}\end{adjustwidth}\endMakeFramed%
}


\theoremstyle{plain}
\newtheorem{theorem}[algorithm]{Theorem}   % Theorems, numbered by chapter
\newtheorem{lemma}[algorithm]{Lemma}         % Lemmas, numbered like theorems
\newtheorem{proposition}[algorithm]{Proposition}  % Propositions, numbered like theorems
\newtheorem{corollary}[algorithm]{Corollary}         % Corollaries, unnumbered

\theoremstyle{definition}            % Bold title, upright body text
\newtheorem{DEFN}[algorithm]{Definition}   % Definitions, numbered like theorems
\newtheorem{EXMP}[algorithm]{Example}      % Examples, numbered like theorems
\newtheorem{problem}[algorithm]{Problem}

\theoremstyle{remark}
\newtheorem{remark}[algorithm]{Remark}      % Remarks, numbered like theorems

% Now framed environments
\newenvironment{definition}{\begin{formal}\begin{DEFN}}{\end{DEFN}\end{formal}}
\newenvironment{example}{\begin{EXMP}\pushQED{\qed}\renewcommand{\qedsymbol}{$\square$}}{\popQED\end{EXMP}}
\renewcommand{\qedsymbol}{$\blacksquare$}

% Figure commands --------------------------------------------------------------

% imgfig ("image figure")
% Shortcut command for inserting an image from the fig/img folder
% Arguments:
%   * (optional) figure width; percentage of text width (default: 0.75)
%   * file name (without fig/img/ and file extension); also used for label
%   * the figure caption
% Examples:
%   \imgfig{image-name}{Caption goes here}
%   \imgfig[0.5]{image-name}{Caption goes here}
\newcommand{\imgfig}[3][0.75]{
  \begin{figure}[htbp]
    \centering
    \includegraphics[width=#1\textwidth]{fig/img/#2}
    \caption{#3}
    \label{fig:#2}
  \end{figure}
}

% dimgfig ("double image figure")
% Shortcut command for inserting two images side by side
% Arguments:
%   * (optional) width split ratio (default: 0.5, i.e. even split)
%   * file name for the left figure, without fig/img/ and file extension
%   * caption for the left figure
%   * file name for the right figure, without fig/img/ and file extension
%   * caption for the right figure
% Examples:
%   \dimgfig{img1}{First caption}{img2}{Second caption}
%   \dimgfig[0.3]{img1}{First caption}{img2}{Second caption}
% Alteratively, see
% https://en.wikibooks.org/wiki/LaTeX/Floats,_Figures_and_Captions#Subfloats
\newcommand{\dimgfig}[5][0.5]{
  \ifx\dimgleftwidth\undefined
    \newlength{\dimgleftwidth}
    \newlength{\dimgrightwidth}
  \fi
  \setlength{\dimgleftwidth}{#1\textwidth-0.02\textwidth}
  \setlength{\dimgrightwidth}{0.96\textwidth-\dimgleftwidth}
  \begin{figure}[htbp]
    \centering
    \begin{minipage}[t]{\dimgleftwidth}
      \centering
      \includegraphics[width=\linewidth]{fig/img/#2}
      \caption{#3}
      \label{fig:#2}
    \end{minipage}
    \hfill
    \begin{minipage}[t]{\dimgrightwidth}
      \centering
      \includegraphics[width=\linewidth]{fig/img/#4}
      \caption{#5}
      \label{fig:#4}
    \end{minipage}
  \end{figure}
}

% Adds line comments to algorithmic environments
\makeatletter
\algnewcommand{\LineComment}[1]{\Statex \hskip\ALG@thistlm $\triangleright$ #1}
\makeatother

\newenvironment{amatrix}[1]{%
  \left[\begin{array}{@{}*{#1}{c}|c@{}}
}{%
  \end{array}\right]
}
