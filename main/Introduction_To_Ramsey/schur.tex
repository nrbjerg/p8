\section{Schur's Theorem}
In this chapter, we will answer the question: ``Given a $r$-coloring on $\mathbb{N}^+$, does there always exist a monochromatic set $\left\{x_1, x_2, \ldots, x_{k}, \sum_{i = 0}^k x_i\right\} \subseteq \mathbb{N}^+$''. To get a better intuition for the problem, consider the case where $r = 3$ and $k = 2$ and $\mathbb{N}^+$ is colored by:
\begin{equation*}
	\chi(n) = \begin{cases} red   & \text{ if } n \equiv 0 \mod 3  \\
              blue  & \text{ if }  n \equiv 1 \mod 3 \\
              green & \text{ otherwise }
	\end{cases}
\end{equation*}
A visual representation of $\chi$ would be: $\textcolor{blue}{1}, \textcolor{green}{2}, \textcolor{red}{3}, \textcolor{blue}{4}, \textcolor{green}{5}, \textcolor{red}{6}, \ldots$ meaning one example of a $red$-monochromatic subset of the form $\left\{x, y, x + y\right\}$, under $\chi$, would be $x = y = 3$, notice that we do not require $x$ and $y$ to be distinct.

Next we will show that the proposition mentioned in the begining of the section does indeed hold.
\begin{theorem}[Schur's Theorem]\label{thm:schur}
	Let $r, k$ be positive integers, then there exists a least positive integer $S(r, k)$ such that any $r$-coloring $\chi: [1, S(r, k)] \to C$ there exists a monochromatic subset of $[1, S(r, k)]$ of the form $\left\{x_1, x_2, \ldots, x_{k}, \sum_{i = 1}^k x_i\right\}$.
\end{theorem}
\begin{proof}
	We will show that any $r$-coloring $\chi$ of $[1, R(k + 1; r)]$ emits a monochromatic subset of the form $\left\{x_1, x_2, \ldots, x_{k}, \sum_{i = 1}^k x_i\right\}$.

	Let $G$ be the complete graph with vertex set $[1, R(k + 1; r) + 1]$. We will define an $r$-edge coloring on $G$ by defining $\chi': E(G) \to C$ as $\chi'(\left\{a, b\right\}) := \chi(\abs{a - b})$, by the definition of $R(k + 1; r)$, the edge coloring $\chi'$ must emit a monochromatic clique $G'$ of order $k + 1$. Next if we order the verticies $v_1, v_2, \ldots v_{k + 1}$ in $G'$ in increasing order, meaning $v_1 < v_2 < \ldots < v_{k + 1}$.
	Then, since $G'$ is monochromatic, we see that
	\begin{equation*}
		\chi(v_i - v_j) = \chi'(\{v_i, v_j\}) = \chi'(\{v_{i'}, v_{j'} \}) = \chi(v_{i'} - v_{j'})
	\end{equation*}
	for all $i > j$ and $i' > j'$, since the verticies are ordered in increasing order. The rest follows by setting $x_j := v_{j + 1} - v_{j}$.
\end{proof}
Let $\chi_{0}: \mathbb{N}^+ \to C$ be an $r$-coloring then, by Schur's Theorem \ref{thm:schur}, there exists a monochromatic subset $A_0 := \left\{x_1, x_2, \ldots, x_{k}, \sum_{i = 1}^k x_i\right\}$ of $[1, S(r, k)]$. Next we may define a new coloring $\chi_1$, which colors every element in $\mathbb{N}^+ \setminus A_{0}$ the same color as $\chi_{0}$, but each element in $A_{0}$ a distinct new color, hence $\chi_1$ is at most a $(r + k + 1)$-coloring, which similarly admits a monochromatic subset $A_1 := \left\{y_1, y_2, \ldots, y_{k}, \sum_{i = 1}^k y_{i}\right\}$ of $[1, S(r + k + 1, k)]$. Additionally we note that $\chi_1(n) = \chi_0(n)$ for each $n \in A_1$ since each element in $A_0$ is colored a new distinct color, by $\chi_{1}$. Hence $A_1$ is also monochromatic under $\chi_0$. Repeating this argument we obtain the following corollary:
\begin{corollary}
	Let $r, k$ be positive integers, any $r$-coloring $\chi: \mathbb{N}^{+} \to \left\{c_1, c_2, \ldots, c_{r}\right\}$ emits infinitely many monochromatic subsets $\mathbb{N}^{+}$ of the form $\left\{x_1, x_2, \ldots, x_{k}, \sum_{i = 1}^k x_i\right\}$.
\end{corollary}


Next we show that statement of Fermats last theorem: ``The equation $x^n + y^n = z^n$, with $n \geq 2$, has no solution $x, y, z \in \mathbb{Z}$, such that $xyz \neq 0$.'' is false if we instead require that $x, y, z$ are elements in some specific family of finite fields.

\begin{theorem}
	Let $n \geq 1$, then there exists a prime $p$ such that for all primes $q \geq p$, the equation $x^n + y^n = z^{n}$ has a solution $x, y, z \in \mathbb{F}_{q}$ with $xyz \neq 0$.
\end{theorem}
\begin{proof}
	Let $q > S(n, 2)$ be a prime, we will consider the multiplicative group $\mathbb{F}_q^{*}$, and the subgroup $G = \left\{x^n \mid x \in \mathbb{F}_p\right\}$.

	Hence there exists $a_1, a_2, \ldots a_k \in \mathbb{F}_q^{*}$ such that $\mathbb{F}_q^{*} = \bigcup_{i = 1}^k a_i G$ with $k = \frac{n}{\gcd(n, p - 1)}$ \textcolor{red}{\textbf{TODO}}. Since $\abs{\mathbb{F}_q^{*}} = \abs{\mathbb{F}_q^{*} / S}\abs{S}$ (lagrange index theorem) and

	Next, we define a $k$-coloring $\chi: \mathbb{F}_q^* \to [k]$ by $\chi(y) = j$ if and only if $y \in a_j G$. Now since $k \leq n$ and $p - 1 \geq S(n, 2)$, there exists a monochromatic triple $\left\{x, y, z\right\} \subseteq \mathbb{F}_q^{*}$ such that $x + y = z$, by Theorem \ref{thm:schur}. Meaning there exists an index $j \in [k]$ such that $a_{j}x^{n}, a_jy^{n}, a_{j}z^{n} \in a_jS$ with $a_{j}x^{n} + a_{j}y^{n} = a_jz^{n}$, the rest follows by multiplying by $a_{j}^{-1}$.
\end{proof}
