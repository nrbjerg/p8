\chapter{Introduction to Ramsey Theory}
%Suppose we have some arbitrary finite coloring $\chi$ on some structure mathematical structure $S$, that is some function $\chi: S \to C$, where $C$ is some set of colors. Given a family of substructures $\mathcal{S}$ of $S$ does $\chi$ admit a monochromatic element in $\mathcal{S}$, provided $S$ is sufficiently large? Ramsey theory is the study of exactly these types of mathematical problems,
Consider some mathematical structure $S$ and a family $\mathcal{F}$ of substructures of $S$. Does every finite coloring $\chi$ on $S$, that is every function $\chi: S \to C$ where $C$ is a finite set of colors, admit a monochromatic element in $\mathcal{F}$, provided $S$ is sufficiently large and if so how large does $S$ have to be? Ramsey theory is the study of exactly these kinds of questions. The theory is quite vast, for instance it includes results on Euclidian geometry and ergrodic theory. So naturally we will focus on the areas of Ramsey theory which are relevant to the field of discrete mathematics. More explicitly we will consider graph Ramsey theory and Ramsey theory over $\mathbb{N}^{+}$, in Chapters \ref{chap:graph_ramsey} and \ref{chap:partition_regularity} respectively.

Chapter \ref{chap:graph_ramsey} starts by establishing Ramsey's theorem, which states that given $\ell_1, \ell_2, \ldots, \ell_{r} \in \mathbb{N}^+$ there exists a least natural number $n = R(\ell_1, \ell_2, \ldots, \ell_{r})$, called a Ramsey number, such that every $r$-coloring $\chi: E(K_n) \to \left\{c_1, c_2, \ldots, c_{r}\right\}$ admits some clique $\mathcal{C}$ of order $\ell_i$ with the edges in $\mathcal{C}$ being colored $c_i$ by $\chi$. In Sections \ref{sec:upper_bound} and \ref{sec:lower_bound} we prove some lower and upper bounds on $R(\ell_1, \ell_2, \ldots, \ell_{r})$. This is followed by Section \ref{sec:exact_values} which establishes the exact values of some small Ramsey numbers. Finally in Section \ref{sec:ass_ramsey} we study the asymptotic behaviors of certain types of Ramsey numbers.

Chapter \ref{chap:partition_regularity} is divided into three main sections each of which dedicated to a distinct theorem from Ramsey theory over $\mathbb{N}^{+}$:
\begin{enumerate}[label=\arabic*.]
	\item The first section covers van der Waerden's Theorem \ref{thm:van_der_waerden}, which asserts that for every $r, k \in \mathbb{N}^+$ there exists a least natural number $W(k, r)$ such that every $r$-coloring of $[1; W(k, r)]$ admits a monochromatic arithmetic progression, that is a set of the form $\left\{a, a + d, \ldots, a + (k- 1)d\right\}$ for some $a, d \in \mathbb{N}^{+}$. Additionally we study a lower bound, constructed using the theory of finite fields, originally proposed by Berlekamp.
	\item The next section covers Schurs Theorem \ref{thm:additive_schur}, which state that there exists a least natural number $S(k, r)$ such that every $r$-coloring of $[1; S(k, r)]$ admits a monochromatic set of the form $\left\{x_1, x_2, \ldots, x_{k}, \sum^k_{i = 1} x_i\right\}$. One of the neat results which follows from Schurs Theorem, is that for each $n \in \mathbb{N}^{+}$ there exists a prime $p$ such that for all primes $q \geq p$, the equation $x^{n} + y^{n} = z^{n}$ has a non-trivial solution in $\mathbb{F}_q$. We conclude this section by studying the asymptotics of $S(2, r)$.
	\item The final section is on Rados Theorems \ref{thm:single_eq_rado} and \ref{thm:rado_full}, we provide a proof of his single equation theorem (Theorem \ref{thm:single_eq_rado}), which classifies which homogeneous linear equations are guarantied\footnote{In the sense that it holds for every finite coloring of $\mathbb{N}^{+}$.}have monochromatic solutions in $\mathbb{N}^+$. From this we are also able to completely characterize which non-homogeneous equations are guarantied to have monochromatic solutions.
\end{enumerate}
