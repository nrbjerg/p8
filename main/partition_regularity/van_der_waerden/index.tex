\section{Van Der Waerden's Theorem}
In this chapter we will study a classical theorem of van der Waerden, concerning $r$-colorings on the integers and arithmetic progressions. Our treatment is based upon the treatment found in \cite{rtoi}[Chapter 2].

\begin{definition}
	Let $a \in \mathbb{Z}$ and $d \in \mathbb{N}^{+}$, the set $\left\{a, a + d, \ldots, a + (k - 1)d\right\}$ is called a \textit{$k$-term arithmetic progression} with \textit{gap} $d$. Let $D \subseteq \mathbb{N}^{+}$, then we will let $AP_D$ denote the family of arithmetic progressions, with gaps $d \in D$.
\end{definition}
That is a $k$-term arithmetic progression is a constant configuration. Next we will present the statement of van der Waerden's theorem, we will delay the proof of the theorem until later in the chapter, in Section \ref{sec:van_der_waerden_proof}.

\begin{theorem}[van der Waerden]\label{thm:van_der_waerden}
	Let $k, r \in \mathbb{N}^{+}$. Then there exists a least natural number $W(k;r)$ such that any $r$ coloring of $[1; W(k;r)]$ admits a monochromatic $k$-term arithmetic progression.
\end{theorem}
To get a better felling for the statement of the theorem we consider the following example.
\begin{example}\label{exmp:Van_Der_Waerden_2_2}
	Consider the case where $k = 2$ and $r \geq 2$, is arbitrary.
	That is we wish to the least natural number $w$ such that any $r$-coloring $\chi: [1; w] \to C$ admits a $2$-term monochromatic arithmetic progression.\\
	Clearly there exists an $r$-coloring of $[1; r]$ that admits no $2$-term monochromatic arithmetic progression, afterall each element in $[1; r]$ may be assigned a distinct color.\\
	Next consider an $r$-coloring $\chi$ on $[1; r + 1]$, by the generalized pigeon hole principle (Theorem \ref{thm:gpp}) there must exist at least two elements $a, b \in [1; r +  1]$ such that $\chi(a) = \chi(b)$, without loss of generalization we may assume that $a < b$.
	The rest follows by setting $d = b - a$ as $\{a, a + d = b\}$ forms a monochromatic $2$-term arithmetic progression.
\end{example}

Only a few small values of $W(k; r)$ are known\footnote{Atleast to the knowledge of the author.}, namely $W(3; 2) = 9, W(3; 3)= 27, W(3; 4) = 76, W(4; 2)= 35, W(4, 3) = 293, W(5; 2) = 178$ and $W(2; 6) = 1132$ confer \cite{van_der_waerden_lower_bound}[Section 6]. Additionally we note that $W(2; r) = r + 1$ for all $r \geq 2$, by Example \ref{exmp:Van_Der_Waerden_2_2}.

The following theorem is based upon \cite{rtoi}[Lemma 4.6 and Theorem 4.9], before we state the theorem we will need some more notation. Let $D \subseteq \mathbb{N}^{+}$ be a set of gaps, then we will let $W^{*}(AP_D, k; r)$ be the least natural number such that every $r$-coloring of $[1; W^{*}(AP_D, k; r)]$ yields a monochromatic $k$-term arithmetic progression, with a gap in $D$, provided such an integer exist. If no such natural number exist we will use the convention that $W^{*}(AP_D, k; r) = \infty$.
%when ever no such integer exists, so whenever $w^{*}(AP_D, k; r)$ is finite we know that each $r$-coloring of $\mathbb{N}^{+}$ admits a monochromatic $k$-term arithmetic progression of with a gap in $D$.

\begin{example}\label{exmp:van_der_waerden_strengthend_version}
	It is relatively easy to find subsets $D$ of $\mathbb{N}^{+}$ such that $W^{*}(AP_D, k; r) = \infty$, for sufficiently large $r$. One general example is when $D$ is any finite subset of $\mathbb{N}^{+}$, say $D = \left\{d_1, d_2, \ldots, d_{m}\right\}$ then $W^{*}(AP_D, k; r) = \infty$ if $r \geq \max D + 1$. This can be seen as follows define an $r$-coloring $\chi: \mathbb{N^{+}} \to \left\{c_0, c_1, \ldots, c_{r - 1}\right\}$ by defining:
	\begin{equation*}
		\chi(n) := c_{[n]_{r}}
	\end{equation*}
	This $r$-coloring $\chi$ admits no monochromatic $k$-term arithmetic progressions with gaps in $D$ since $\chi(n) = c$ implies $\chi(n + d_{j}) \neq c$ for each $j \in [1; m]$.
\end{example}

\begin{theorem}[Strengthened van der Waerden] \label{thm:strengthend_version_of_van_der_waerden}
	Let $D \subseteq \mathbb{N}^{+}$, $r, k \geq 2$, and $m \in \mathbb{N}^{+}$, then
	\begin{equation}\label{eq:strengthend_van_der_waerden_1}
		W^{*}(AP_{mD}, k; r) = m(W^{*}(AP_D, k; r) - 1) + 1
	\end{equation}
	with the usual conventions for addition and multiplications involving $\infty$, and in particular:
	\begin{equation}\label{eq:strengthed_van_der_waerden_2}
		W^{*}(AP_{m\mathbb{N}^{+}}, k; r) = m(W(k; r) - 1) + 1
	\end{equation}
	for all $m, k, r \in \mathbb{N}^{+}$ such that $k, r \geq 2$.
\end{theorem}
\begin{proof}
	To prove that Equation \eqref{eq:strengthend_van_der_waerden_1} holds, we consider two disjoint cases:
	\begin{enumerate}
		\item First consider the case where $W^{*}(AP_D, k; r) < \infty$. We will start by proving that:
		      \begin{equation*}
			      W^{*}(AP_{mD}, k; r) \leq m(W^{*}(AP_D, k; r) - 1) + 1
		      \end{equation*}
		      Thus, let $\chi$ be any $r$-coloring of $[1; m(W^{*}(AP_D, k; r) - 1) + 1]$, then we may define an $r$-coloring $\chi'$ on $[1; W^{*}(AP_D, k; r)]$ as $\chi'(n) := \chi(m(n - 1) + 1)$.
		      However by our assumption on $W^{*}(AP_D, k; r)$, the $r$-coloring $\chi'$ admits a monochromatic $k$-term arithmetic progression $\left\{a, a + d, \ldots, a + (k - 1)d\right\}$ with $d \in D$.
		      However by the definition of $\chi'$ the sequence $\left\{m(a - 1) + 1, m(a + d - 1) + 1, \ldots, m(a + (k - 1)d - 1) + 1\right\}$ is monochromatic under $\chi$, furthermore it is easy to see that this sequence is a $k$-term arithmetic progression with a gap $md \in mD$.

		      Next we will show that:
		      \begin{equation*}
			      W^{*}(AP_{mD}, k; r) \geq m(W^{*}(AP_D, k; r) - 1) + 1
		      \end{equation*}
		      Let $\gamma$ be an $r$-coloring of $[1; W^{*}(AP_D, k; r) - 1]$, which does not admit a monochromatic $k$-term arithmetic progression with a gap in $D$.
		      We may construct an $r$-coloring $\gamma'$ on $[1; m(W^{*}(AP_D, k; r) - 1)]$ which does not admit a monochromatic $k$-term arithmetic progression with a gap in $mD$, by defining $\gamma'$ as $\gamma'(n) = \gamma(n')$ if and only if $n \in [m(n' -1) + 1; mn']$.
		      Then $\gamma'$ does not admit a monochromatic $k$-term arithmetic progression with a gap in $mD$, since $d \in D$ and $a < m(W^{*}(AP_D, k; r) - 1) - md$ implies $\gamma'(a) \neq \gamma'(a + md)$, since $d \geq 1$.
		\item Finally if $W^{*}(AP_D, k; r) = \infty$, then for every $K \in \mathbb{N}$ there exists a $r$-coloring $\psi$ of $[1; K]$ such that $\psi$ admits no $k$-term arithmetic progression with a gap $d \in D$. We may construct an $r$-coloring $\psi'$ of $[1; m(K - 1) + 1]$, from $\psi$ analogously to how $\gamma'$ was constructed from $\gamma$, which admits no $k$-term arithmetic progression with a gap in $mD$.
	\end{enumerate}
	Finally Equation \eqref{eq:strengthed_van_der_waerden_2} follows directly by Equation \eqref{eq:strengthend_van_der_waerden_1} by setting $D = \mathbb{N}^{+}$, since $W^{*}(AP_{\mathbb{N}^+}, k; r) = W(k; r)$.
\end{proof}

\newpage
\subsection{A Proof of Van Der Waerdens Theorem}\label{sec:van_der_waerden_proof}
Finally we will provide a proof of van der Waerdens Theorem \ref{thm:van_der_waerden}. First we will need some results based on those found in \cite{rtoi}[Section 2.6 and Exercise 2.15]

\begin{proposition}\label{prop:2.30_rtoi}
	Let $a, b \in \mathbb{N}^+$ and $\mathcal{F}$ be a family of subsets of $\mathbb{N}^{+}$ such that $S \in \mathcal{F}$ if and only if $(a - b) + bS = \left\{a + b(s - 1) | s \in S\right\} \in \mathcal{F}$. Let $r \in \mathbb{N}^+$ then every $r$-coloring of $[1, n]$ yields a monochromatic member of $\mathcal{F}$ if and only if every $r$-coloring of
	\begin{equation*}
		a + b[0; n - 1] = \left\{a, a + b, \ldots, a + (n - 1)b\right\}
	\end{equation*}
	yields a monochromatic member of $\mathcal{F}$.
\end{proposition}
\begin{proof}
	``$\implies$'' Let $\chi$ be an $r$-coloring on $a + b[0; n - 1]$, then we define an $r$-coloring $\chi'$ on $[1;n]$ as:
	\begin{equation*}
		\chi'(k) = \chi(a + b(k - 1))
	\end{equation*}
	then $\chi'$ admits a $S \in \mathcal{F}$, however this means that $(a - b) + bS$ is monochromatic under $\chi$ and is a subset of $a + b[0; n - 1]$. \\
	``$\impliedby$'' Let $\chi$ be an $r$-coloring on $[1; n]$, then we define an $r$-coloring $\chi'$ on $a + b[0; n - 1]$ as:
	\begin{equation*}
		\chi'(a + bk) = \chi(k + 1)
	\end{equation*}
	now since $\chi'$ is an $r$-coloring on $a + b[0; n - 1]$, there exists a monochromatic subset $S \in \mathcal{F}$ of $a + b[0; n - 1]$, but then $S = (a - b) + bS'$, for some $S' \in \mathcal{F}$ by our condition on $a$ and $b$. Hence $S'$ is a monochromatic subset under $\chi$.
\end{proof}
We note that the family $\mathcal{F}$ described in Proposition \ref{prop:2.30_rtoi} can alternatively be thought of as constant configurations over $\mathbb{N}^{+}$.

%\begin{proposition}\label{prop:2.30_rtoi}[original version]
%	Let $a, b \in \mathbb{N}^+$ and $\mathcal{F}$ be a family of subsets of $\mathbb{N}^{+}$ such that $S \in \mathcal{F}$ if and only if $a + bS = \left\{a + bs | s \in S\right\} \in \mathcal{F}$. Let $r \in \mathbb{N}^+$ then every $r$-coloring of $[1, n]$ yields a monochromatic member of $\mathcal{F}$ if and only if every $r$-coloring of
%	\begin{equation*}
%		a + b[0; n - 1] = \left\{a, a + b, \ldots, a + (n - 1)b\right\}
%	\end{equation*}
%	yields a monochromatic member of $\mathcal{F}$.
%\end{proposition}
%
%\begin{proof}[Proof of Proposition \ref{prop:2.30_rtoi} original version]
%	``$\implies$'' \textcolor{red}{\textbf{TODO}} there is something iffy here. Let $\chi: a + b[0; n - 1] \to C$ be an $r$-coloring. Next we define another $r$-coloring $\chi': [1; n] \to C$ by $\chi'(k) = \chi(a + b (k - 1))$, by our assumption this $r$-coloring of $[1; n]$ admits a monochromatic subset $S \in \mathcal{F}$. The rest follows as $a + bS$ must be monochromatic under $\chi$.
%
%	``$\impliedby$'' Let $\chi: [1; n] \to C$ be an $r$-coloring, then we define $\chi': a + b[0; n - 1] \to C$ by $\chi'(a + b k) = \chi(k + 1)$ that is $\chi'(n) = \chi \left( \frac{n - a}{b} + 1\right)$, applying our assumption we see that $\chi'$ admits a monochromatic subset $S \in \mathcal{F}$ of $a + b[0; n - 1]$, however this implies that $(S - a) \cdot \frac{1}{b} \in \mathcal{F}$ is monochromatic under $\chi$, additionally we note that $(S - a) \cdot \frac{1}{b}$ is indeed a subset of $[1; n]$.
%\end{proof}

\begin{definition}
	Let $r,m,n \in \mathbb{N}^{+}$. Let $\gamma: [1; n + m] \to C$ be an $r$-coloring, the \textit{$r^m$-coloring $\chi_{\gamma, m}: [1; n] \to C^{m}$ derived from $\gamma$} is defined as:
	\begin{equation*}
		\chi_{\gamma, m}(i) = (\gamma(i + 1), \gamma(i + 2), \ldots, \gamma(i + m))
	\end{equation*}
\end{definition}
Note that $\chi_{\gamma, m}(i) = \chi_{\gamma, m}(j)$ if and only if $[i + 1; i + m]$ and $[j + 1; i + m]$ is colored in the same way by $\gamma$, that is $\gamma(i + k) = \gamma(j + k)$ for all $k \in [1; m]$.

\begin{definition}\label{def:refined}
	Let $k, t, r \in \mathbb{N}^{+}$ then the triple $(k, t, r)$ is called \textit{refined} if there exists an $m \in \mathbb{N}^{+}$ such that for every $r$-coloring $\chi: [1; m] \to C$, there exists $z, x_0, x_1, \ldots, x_t \in \mathbb{N}^+$ such that every set:
	\begin{equation*}
		T_s := \left\{b_{s} + \sum_{i = 0}^{s - 1} \lambda_i x_i \middle| \lambda_i \in [1; k]\right\} \text{ with }  b_s = z + (k + 1) \sum_{i = s}^t x_i
	\end{equation*}
	is monochromatic for all $s \in [0; t]$.
\end{definition}
\begin{remark}\label{rem:2.34_rtoi}
	The definition of a refined triple $(k, t, r)$ may look quite abstract however if we take $\lambda_0 = \lambda_1 = \cdots = \lambda_{s - 1} = j$ with $j \in [1; k]$ see that $a + jd \in T_{s}$ where $a = b_s$ and $d = \sum_{i = 0}^{s - 1} x_{i}$ hence we have that the arithmetic progression $\left\{a + d, a + 2d, \ldots, a + kd\right\}$ is contained within the monochromatic set $T_{s}$.
\end{remark}
\begin{example}\label{exmp:refined}
	To get a better intuition for what it means for a triple to be refined, we will consider the triple $(2, 2; r)$. In this case the sets $T_s$ using the same notation as in Definition \ref{def:refined} will be of the form\footnote{Note that $z, x_0, x_1, x_2$ depend on the coloring $\chi$, however the general forms of the sets remain the same.}:
	\begin{align*}
		T_0 & = \left\{b_0\right\}                                                                            \\
		T_1 & = \left\{b_1 + x_0, b_1 + 2x_{0}\right\}                                                        \\
		T_1 & = \left\{b_2 + x_0 + x_{1}, b_2 + 2x_{0} + x_{1}, b_2 + x_1 + 2x_2, b_{2} + 2x_1 + 2x_2\right\}
	\end{align*}
	with $b_0 = z + 3 (x_0 + x_1 + x_2), b_1 = z + 3 (x_1 + x_{2})$ and $b_2 = z + 3x_{2}$. Clearly $T_0$ is monochromatic, under all colorings, afterall $\abs{T_0} = 1$. Hence $(2, 2, r)$ is refined if and only if there exists an $m \in \mathbb{N}^+$ such that for any $r$-coloring $\chi$ on $[1; m]$ there exists $z, x_0, x_1, x_2 \in \mathbb{N}^{+}$ with:
	\begin{equation*}
		\chi(b_1 + x_{0}) =  \chi(b_1 + 2x_{0})
	\end{equation*}
	and
	\begin{equation*}
		\chi(b_2 + x_0 + x_1) = \chi(b_2 + 2x_0 + x_1) = \chi(b_2 + x_0 + 2x_1) = \chi(b_2 + 2x_0 + 2x_1)
	\end{equation*}
	Please notice that we do not require that $T_0, T_1, T_2$ to be colored the same color.
\end{example}

In order to prove van der Waerden's theorem we will need two lemmas, which asserts some results regarding refined triples and van der Waerden numbers.
\begin{lemma}\label{lem:2.35_rtoi}
	Let $k \geq 2$, if $W(k; r)$ exists for all $r \in \mathbb{N}^{+}$, then the triple $(k, t, r)$ is refined for all $r, t \in \mathbb{N}^{+}$
\end{lemma}
\begin{proof}
	Let $r \in \mathbb{N}^{+}$, we will prove the lemma use induction on $t$. To prove that $(k, 1, r)$ is refined, we show that we may take $m$ (from Definition \ref{def:refined}) to be $3W(k; r) + k + 1$. Hence consider an arbitrary $r$-coloring $\chi: [1; 3W(k;r) + k + 1] \to C$. We wish to apply Proposition \ref{prop:2.30_rtoi}, by letting $\mathcal{F}$ to be the family of arithmetic progressions. Notice that we may pick $a = W(k;r) + k + 3$ and $b = 1$, since if $S$ is a $k$-term arithmetic progression, then $(a - b) + bS = \left\{W(k; r) + k + 2 + s | s \in S\right\}$, is simply another $k$-term arithmetic progression. Hence we see that the interval $[W(k;r) + k + 2; 2W(k; r) + k + 1]$ must contain a monochromatic $k$-term arithmetic progression\footnote{That is under our assumption that $W(k;r)$ exists.}
	\begin{equation*}
		\left\{a' + d, a' + 2d, \ldots, a' + kd\right\}
	\end{equation*}
	again using the notation of Definition \ref{def:refined}, let $z = a' - (k + 1), x_0 = d$ and $x_1 = 1$. Then $T_{0} = \left\{a' + (k + 1)d\right\}$ and $T_1 = S$ are both monochromatic, under $\chi$, and both are subsets of $[1; 3W(k; r) + k + 1]$. Hence $(k, 1, r)$ is a refined triple.

	Next for the induction step assume that $t \in \mathbb{N}^+$ and that $(k, t, r)$ is refined. We will show that this implies that $(k, t + 1, r)$ is refined. First let $m := m_{k, t, r}$ be as in definition \ref{def:refined} we will show that we may take $m_{k, t + 1; r} := m + 2W(k; r^{m})$. Let $\gamma: [1, m_{k, t + 1; r}] \to C$ be an arbitrary $r$-coloring. Let $\chi_{\gamma, m}$ be the $r^m$-coloring of $[1; 2W(k; r^{m})]$ derived from $\gamma$. This coloring $\chi_{\gamma, m}$ must admit a $k$-term arithmetic progression\footnote{Again under our the assumption that $W(k; r^{m})$ exists}
	\begin{equation*}
		\left\{a' + d, a' + 2d, \ldots, a' + kd\right\} \subseteq [1; 2W(k; r^{m})]
	\end{equation*}
	Since $\chi_{\gamma, m}$ is a derived from $\gamma$, we must have that the $k$ intervals $I_j := [a' + jd + 1, a' + jd  + m], j \in [1; k]$ are colored identically under $\gamma$.
	Additionally we note that since $(k, t, r)$ is refined, there exists $z, x_0, x_1, \ldots, x_t \in \mathbb{N}^{+}$ such that each $T_s$'s (as in Definition \ref{def:refined}) are monochromatic under $\gamma$.
	Hence each interval $I_j$ contains the monochromatic (under $\gamma$) sets:
	\begin{equation*}
		S_s(j) = T_{s} + (a' + jd) = \left\{(b_s + a' + jd) + \sum_{i = 0}^{s - 1} \lambda_ix_i \middle| \lambda_i \in [1; k]\right\}, \text{ with }  s \in [0; t]
	\end{equation*}
	recall that $m = m_{k, t, r}$ and that each $I_j$ was colored the same under $\gamma$. Furthermore since each interval has the same coloring under $\gamma$, we have that $S_s(u)$ and $S_s(v)$ must have the same coloring under $\gamma$ for all $u, v \in [1; k]$. Hence the sets:
	\begin{equation*}
		Q_s = \left\{(b_s + a') + \sum_{i = 0}^{s -  1} \lambda_i x_i + jd \middle| j, \lambda_i \in [1; k]\right\},i \text{ with }  s \in [0; t]
	\end{equation*}
	is monochromatic under $\gamma$. It remains to find $z', x'_0, x'_1, \ldots, x'_{t + 1}$ which produces monochromatic sets (under $\gamma$) $T'_s$ for $s \in [0; t+ 1]$ to show that $(k, t + 1, r)$ is refined. This follows by letting:
	\begin{align*}
		z'   & = z + a',                                  \\
		x_0' & = d                                        \\
		x'_i & = x_{i - 1} \text{ for }  s \in [1; t + 1]
	\end{align*}
	since then, for each $s \in [0; t]$ we have:
	\begin{align*}
		T'_{s + 1} & = \Bigg\{b'_{s + 1} + \sum_{i = 0}^{s} \lambda_i x'_i \Bigg| \lambda_i \in [1; k]\Bigg\}                                                                                             \\
		           & = \Bigg\{z' + (k + 1) \sum_{i = {s + 1}}^{t + 1} x'_i + \sum_{i = 1}^{s} \lambda_i x'_i + \lambda_0 d \Bigg| \lambda_i \in [1; k]\Bigg\}                                             \\
		           & = \Bigg\{\underset{=(b_s + a')}{\underbrace{z + a' + (k + 1) \sum_{i = s}^t x_i}} + \sum_{i = 0}^{s - 1} \lambda_{i + 1} x_i +\lambda_0 d \Bigg| \lambda_i \in [0; k] \Bigg\}  = Q_s
	\end{align*}
	which are monochromatic under $\gamma$. The rest follows as $\abs{T'_0} = 1$ and hence it is trivially monochromatic, since each $Q_s$ is monochromatic under $\gamma$. Hence $(k, t + 1, r)$ is refined.
\end{proof}
\begin{lemma}\label{lem:2.36_rtoi}
	Let $k \in \mathbb{N}^+$, if $(k, t, r)$ is a refined triple for all $t, r \in \mathbb{N}^{+}$, then $W(k + 1, r)$ exists for all $r \in \mathbb{N}^{+}$.
\end{lemma}
\begin{proof}
	Let $r \in \mathbb{N}^{+}$ be arbitrary and $\chi$ be an arbitrary $r$-coloring on $\mathbb{N}^{+}$, by our assumption the triple $(k, r, r)$ is refined. Meaning there exists $z, x_0, x_1, \ldots, x_{r} \in \mathbb{N}^{+}$ such that $T_0, T_1, \ldots, T_{r}$, defined as in Definition \ref{def:refined}, is monochromatic under $\chi$. By the generalized pigeonhole principle (Theorem \ref{thm:gpp}) there exists at least two sets $T_v, T_w$, with $v \neq w$, which $\chi$ colors the same color, that is $\chi(T_v) = \chi(T_w)$. We have:
	\begin{equation*}
		T_v = \left\{ z + (k + 1)\sum_{i = v}^{r} x_{i} + \sum_{i = 0}^{r - 1} \lambda_ix_i \middle| \lambda_i \in [1;k]\right\}
	\end{equation*}
	and
	\begin{equation*}
		T_w = \left\{ z + (k + 1)\sum_{i = u}^{r} x_{i} + \sum_{i = 0}^{r - 1} \lambda_ix_i \middle| \lambda_i \in [1;k]\right\}
	\end{equation*}
	Without loss of generality we may assume that $v < w$, then setting $a = z + \sum_{i = 0}^{v - 1} x_i + (k + 1) \sum_{i = w}^{r} x_{i}$, we may write:
	\begin{equation*}
		T_v = \left\{a + (k + 1) \sum_{i = v}^{w - 1} x_i + \sum_{i = 0}^{v - 1} (\lambda_i - 1) x_i \middle| \lambda_i \in [1; k]\right\}
	\end{equation*}
	and
	\begin{equation*}
		T_w = \left\{a - \sum_{i = 0}^{v - 1} x_i + \sum_{i = 0}^{w - 1} \lambda_i x_i \middle| \lambda_i \in [1; k]\right\}
	\end{equation*}
	Fixing $\lambda_0 = \lambda_1 = \cdots = \lambda_{v - 1} = 1$ we have:
	\begin{equation*}
		T_w' = \left\{a + \sum_{i = v}^{w - 1} \lambda_i x_i \middle| \lambda_i \in [1; k]\right\} \subseteq T_{w}
	\end{equation*}
	Letting $d = \sum_{i = v}^{w - 1} x_i$ we see that $a + (k + 1)d \in T_v$ and by Remark \ref{rem:2.34_rtoi}, we have:
	\begin{equation*}
		\left\{a + d, a + 2d, \ldots, a + kd\right\} \subseteq T_w'
	\end{equation*}
	Hence $\{a + d, a + 2d, \ldots, a + (k + 1)d\}$ forms a monochromatic $(k + 1)$-term arithmetic progression of length $k + 1$ under $\chi$. The fact that $W(k + 1; r)$ exists follows directly by the well ordering principle.
\end{proof}

Now using Lemmas \ref{lem:2.35_rtoi} and \ref{lem:2.36_rtoi} we are finally able to prove van der Waerden's theorem.
\begin{proof}[Proof of van der Waerden's Theorem \ref{thm:van_der_waerden}]
	By Example \ref{exmp:Van_Der_Waerden_2_2} $W(2; r)$ exists for all $r \in \mathbb{N}^+$, hence by Lemma \ref{lem:2.35_rtoi} $(2, t, r)$ is refined for all $r, t$ which in turn implies that $W(3; r)$ exists, by Lemma \ref{lem:2.36_rtoi}. The rest similarly follows by repeated applications of Lemmas \ref{lem:2.35_rtoi} and \ref{lem:2.36_rtoi}.
\end{proof}

\subsection{Bounds on Van Der Waerden Numbers}
During this section we will present some bounds on van der Waerden Numbers, we will focus our efforts on lower bounds of $W(k; r)$, since the upper bounds are generally enormous, for example the best known general upper bound is due to \cite{van_der_waerden_upper_bound} asserts that:
\begin{equation*}
	W(r; k) \leq 2^{2^{r^{2^{2^{k + 9}}}}}
\end{equation*}
for all $k, r \geq 2$. In contrast the following theorem is the best general lower bound for $W(p + 1; 2)$ when $p$ is a prime, the bound is due to \cite{berlekamp_lower_bound}. However our proof will be based on the proof found in \cite{berlekamp_proof}. First we will need some basic results on the properties of field extensions:

\newpage
\begin{lemma}\label{lem:field_extension}
	Let $F / K$ be a field extension and $\alpha \in F \setminus K$ be algebraic over $K$ with minimal polynomial $f \in K[X]$. Then the following assertions holds:
	\begin{enumerate}
		\item Let $g \in K[X]$ with $g(\alpha) = 0$, then $f | g$. \label{lem:field_extension1}
		\item $K(\alpha) = K[\alpha] \cong K[X] / \gen{f}$. \label{lem:field_extension2}
	\end{enumerate}
\end{lemma}
\begin{proof}
	We start by proving that Assertion \ref{lem:field_extension1}, by polynomial division there exists $q, r \in K[X]$ with $\deg(r) < \deg(f)$ such that $g = fq + r$. Assume for the sake of contradiction that $r \neq 0$, then we have $r(\alpha) = g(\alpha) - f(\alpha)q(\alpha) = 0$, clearly a contradiction since this would imply that $f$ is not a minimal polynomial of $\alpha$.

	Next to prove Assertion \ref{lem:field_extension2} consider the function $\ev_{\alpha}: K[X] \to K[\alpha]$ defined as:
	\begin{equation*}
		\ev_{\alpha}(f) = f(\alpha)
	\end{equation*}
	Note that $\ev_{\alpha}$ is a surjective homomorphism, and that by Assertion \ref{lem:field_extension1} we have:
	\begin{equation*}
		\ker(\ev_{\alpha}) = \left\{g \in K[X] \middle| f \text{ divides }  g\right\} = \gen{f}
	\end{equation*}
	Thus by the ring isomorphism theorem we have that $K[X] / \gen{f} \cong K[\alpha]$. Next since $f$ is irreducible\footnote{If $f$ is not irreducible, then $\alpha$ must be a root of one of its factors, contradicting the fact that $f$ is a minimal polynomial of $\alpha$} we have that $K[X] / \gen{f}$ is a field, which in turn implies that $K(\alpha) = K[\alpha]$, since $K[\alpha] \subseteq K(\alpha)$.
\end{proof}

\begin{lemma}\label{lem:no_roots}
	Let $p$ be a prime and $g$ a primitive element of $\mathbb{F}_{2^p}$ and $f \in \mathbb{F}_{2}[X]$ with $\deg(f) \leq p - 1$, then $f(g^k) \neq 0$ for all $k \in [1; 2^{p} - 1]$.
\end{lemma}
\begin{proof}
	First notice that there exists no intermediate field $F$ between $\mathbb{F}_2$ and $\mathbb{F}_{2^{p}}$, that is a field $F$ such that $\mathbb{F}_{2} \subset F \subset \mathbb{F}_{2^p}$, since this would imply that:
	\begin{equation*}
		[\mathbb{F}_{2^p} : F] [F : \mathbb{F}_2] = [\mathbb{F}_2 : \mathbb{F}_{2^p}] = p
	\end{equation*}
	with $[\mathbb{F}_{2^p} : F] \neq 1$ and $[F : \mathbb{F}_2] \neq 1$.

	Consider an arbitrary element $g^k$ with $k \in [1; 2^p - 1]$, then $g^k \in \mathbb{F}_{2^p} \setminus \mathbb{F}_2$, notice that this means that $\mathbb{F}_2(g^k) = \mathbb{F}_{2^p}$, since there exists no intermediate fields. Additionally since $\mathbb{F}_{2^p} / \mathbb{F}_2$ is a finite and hence algebraic field extension, we see that $g^k$ has a minimal polynomial $h \in \mathbb{F}_2[X]$. By Lemma \ref{lem:field_extension}\ref{lem:field_extension2}, we must have $\deg(h) = p$, since $\mathbb{F}_{2^p} \cong \mathbb{F}_2[X] / \gen{h}$. The rest follows by Lemma \ref{lem:field_extension}\ref{lem:field_extension1}.
\end{proof}

\begin{theorem}[Berlekamps Lower Bound]\label{thm:berlekamps_lower_bound}
	Let $p$ be a prime, then $W(p + 1; 2) > p (2^{p} - 1)$.
\end{theorem}
\begin{proof}
	Consider the finite field $\mathbb{F}_{2^p}$ with $2^p$ elements, and let $g$ be a primitive element of $\mathbb{F}_{2^p}^{*}$. Fixing a $\mathbb{F}_{2}$ basis $v_1, v_2, \ldots, v_p \in \mathbb{F}_{2^p}$ of $\mathbb{F}_{2^{p}}$, we may write:
	\begin{equation}\label{eq:lin_comb}
		g^j = \sum_{i = 1}^p a_{i,j} v_i \text{ for } j \in [1; p(2^p - 1)]
	\end{equation}
	where $a_{i, j} \in \mathbb{F}_2$. We claim that the coloring $\chi: [1; p(2^p - 1)] \to \{c_0, c_1\}$, defined as $\chi(j) = c_{a_{1, j}}$ does not yield a monochromatic $(p + 1)$-term arithmetic progression. Assume for the sake of contradiction that $\chi$ does yield a monochromatic $(p + 1)$-term arithmetic progression, say:
	\begin{equation*}
		\left\{a, a + d, \ldots, a + pd\right\} \subseteq [1; p(2^p - 1)]
	\end{equation*}
	Next we will let $\alpha = g^{a}$ and $\beta = g^{d}$, note that since $a + pd \leq p(2^p - 1)$ and $a \geq 1$ we see that $d \leq 2^p - 2$, thus $\beta \neq 1$. Consider the set $A := \left\{g^a, g^{a + d}, \ldots, g^{a + pd}\right\} = \left\{\alpha, \alpha \beta, \ldots, \alpha \beta^{p}\right\}$
	We will consider the cases where the monochromatic $(p + 1)$-term arithmetic progression is colored $c_0$ and $c_1$ separately.
	\begin{enumerate}
		\item If $\chi\left(\left\{a, a + d, \ldots, a + pd\right\}\right) = \{c_0\}$, then all of the $p + 1$ elements of $A$ must be contained within the $p - 1$ dimensional space $\Span_{\mathbb{F}_2} \left\{v_2, \ldots, v_{p}\right\}$. Hence any subset of $A$ containing $p$ elements are linearly dependent over $\mathbb{F}_2$. In particular there exists $\lambda_0, \lambda_1, \ldots \lambda_{p - 1} \in \mathbb{F}_2$, not all $0$, such that:
		      \begin{equation*}
			      \sum_{i = 0}^{p - 1} \lambda_i \alpha \beta^{i} = 0
		      \end{equation*}
		      however $g$ is a primitive element of $\mathbb{F}_{2^p}$ we must have $\alpha \neq 0$ meaning we must have that $\sum_{i = 0}^{p - 1} \lambda_i \beta^i = 0$, which is a contradiction by Lemma \ref{lem:no_roots}.
		\item Conversely if $\chi\left(\left\{a, a + d, \ldots, a + pd\right\}\right) = \{c_{1}\}$, then consider the set $I - \alpha = \left\{0, \alpha(\beta - 1), \ldots, \alpha(\beta^p - 1)\right\}$, this set is once again contained within $\Span_{\mathbb{F}_2} \left\{v_2, \ldots, v_p\right\}$\footnote{This can be seen by expanding each element in $I$ and $\alpha$ via the linear combination in Equation \eqref{eq:lin_comb}}, meaning there exists $\lambda_0, \lambda_1, \ldots, \lambda_{p - 1} \in \mathbb{F}_2$, not all $0$ such that:
		      \begin{equation*}
			      \sum_{i = 0}^{p - 1} \lambda \alpha (\beta^i - 1) = 0
		      \end{equation*}
		      once again since $\alpha \neq 0$, we see that $\sum_{i = 1}^{p - 1} \lambda_i \beta^{i} - \sum_{i = 0}^{p - 1} \lambda_{i} = 0$, which once again yields a contradiction by Lemma \ref{lem:no_roots}. \qedhere
	\end{enumerate}
\end{proof}

Next discuss a generalization of Berlekamps lower bound, presented in the paper \cite{van_der_waerden_lower_bound}. In order to prove this generalization we will use the following theorem,
\begin{theorem}\label{thm:from_paper}
	Let $r, k \geq 2$, then $W(k; r) > p \left(w \left(k; r - \ceil{\frac{r}{p}}\right) - 1\right)$, where $p$ is the largest prime such that $p \leq k$.
\end{theorem}
We will not provide a proof of the theorem instead we refer to \cite{van_der_waerden_lower_bound}[Section 2]. However we note that the general idea of the proof is to ``blow-up'' a $r - \ceil{\frac{r}{p}}$-coloring $\chi$ on $\left[1; w \left(k; r - \ceil{\frac{r}{p}}\right) - 1\right]$, which admits no monochromatic $k$-term arithmetic progressions, to get a $r$-coloring of $\left[1; p \left(w \left(k; r - \ceil{\frac{r}{p}}\right) - 1\right) \right]$, which similarly admits no monochromatic $k$-term arithmetic progressions. Their approach is similar to the approach used in proof of Proposition \ref{prop:r_ell_k_is_super_multiplicative}, however instead of replacing vertices with copies of an edge colored graphs, they replace each $n \in \left[1; w \left(k; r - \ceil{\frac{r}{p}}\right) - 1\right]$ with a finite colored subset of $\mathbb{N}^{+}$ of cardinality $p$.

%This leads us to the following easy corollary, which generalizes Berlekamps lower bound (Theorem \ref{thm:berlekamps_lower_bound}).
\newpage
\begin{corollary}[Berlekamps Lower Bound Generalized]
	Let $r \geq 2$ and $p$ be any prime, such that $r \leq p$, then:
	\begin{equation*}
		W(r; p + 1) > p^{r - 1}(2^{p} - 1)
	\end{equation*}
\end{corollary}
\begin{proof}
	We prove the corollary using induction on $r$, the case where $r = 2$, is covered by Berlekamps lower bound (Theorem \ref{thm:berlekamps_lower_bound}). Next assume $W(p + 1; r) > p^{r - 1}(2^{p} - 1)$, then
	\begin{align*}
		W(p + 1; r + 1) & \stackrel{(a)}{>}p \left[w\left(p + 1; r - \ceil{\frac{r}{p}}\right) - 1\right]  \\
		                & \stackrel{(b)}{=} p \left[W(p + 1; r - 1) - 1\right]                             \\
		                & \stackrel{(c)}{\geq} p \left[p^{r - 1}(2^{p} - 1) + 1 - 1\right] = p^r (2^p - 1)
	\end{align*}
	where $(a)$ follows by Theorem \ref{thm:from_paper}, with $k = p + 1$, $(b)$ follows as $r \leq p$ and hence $\ceil{\frac{r}{p}} = 1$ and $(c)$ from the induction hypothesis.
\end{proof}

\subsection{Arithmetic Progressions $(\textnormal{mod } m)$}
Before we move on from the topic of arithmetic progressions and in particular van der Waerdens Theorem, we will briefly consider a different type of arithmetic progression.

\begin{definition}
	Let $m \geq 2$, the strictly increasing sequence $\left\{x_i\right\}_{i = 1}^k$ with $x_i \in \mathbb{N}^{+}$ is called a \textit{$k$-term $(\textit{mod } m)$-arithmetic progression} if there exists an $a \in [1; m - 1]$ such that $x_{i + 1} - x_{i} \equiv a \mod m$ for all $i \in [1; k - 1]$. We will denote the family of $k$-term $(\textnormal{mod } m)$-arithmetic progressions by $AP_{(m, k)}$ and the family:
	\begin{equation*}
		AP_{(m)} := \bigcup_{k \in \mathbb{N}^{+}} AP_{(m, k)}
	\end{equation*}
	will be refereed to as \textit{family of $(\textit{mod } m)$-arithmetic progressions}.
\end{definition}
\begin{remark}
	Notice that we do not allow $a$ to equal $0$ or equivalently $m$, if we did allow this would have $AP_{\left\{m\right\}} \subseteq AP_{(m)}$.
\end{remark}

\begin{example}\label{exmp:mod_arithmetic_progression}
	The sequence $\left\{8, 17, 34, 43, 68\right\}$, forms a $5$-term $(\textnormal{mod } 8)$-arithmetic progression. Notice that we do not require that the gap between consecutive numbers in the arithmetic progression to be fixed, but rather we simply require that the gap between consecutive numbers are is in the same residue class modulo $8$.
\end{example}

Contrary to the partition regularity of the family $k$-term arithmetic progressions, by van der Waerdens Theorem \ref{thm:van_der_waerden}, and even the partition regularity of $k$-term arithmetic progressions in $AP_{m\mathbb{N}^{+}}$, by Theorem \ref{thm:strengthend_version_of_van_der_waerden}, we have that the family $AP_{(m, k)}$ is not partition regular, provided $k$ is sufficiently large with respect to $m$. We will prove this in the following theorem:

\begin{theorem}
	Let $m \geq 2$ and $k > \ceil{\frac{m}{2}}$. Then the family $AP_{(m, k)}$ is not partition regular.
\end{theorem}
\begin{proof}
	%We may assume that $k = \ceil{\frac{m}{2}}$, since it is sufficient to show that the family of $\ceil{\frac{m}{2}}$-term $(\textnormal{mod } m)$-arithmetic progressions aren't partition regular\footnote{Since every $k$-term (\textnormal{mod m})-arithmetic progression, with $k \geq \ceil{\frac{m}{2}}$, must contain a $\ceil{\frac{m}{2}}$-term $(\textnormal{mod } m)$-arithmetic progression.}.
	For the sake of simplicity we will let $M := \ceil{\frac{m}{2}}$ throughout the proof. To prove the theorem we will consider that the following $2$-coloring $\chi: \mathbb{N} \to \left\{red, blue\right\}$ defined as:
	\begin{equation*}
		\chi(n) = \begin{cases}
			red  & \text{ if } [n]_{M} < M \\
			blue & \text{ otherwise }
		\end{cases}
	\end{equation*}
	where $[n]_{M}$ denotes the remainder of $n$ after division by $M$. We will show that if $\chi$ admits a $K$-term $(\textnormal{mod } m)$-arithmetic progression then we must have $K \leq M$.

	Next, fix an arbitrary $a \in [1; k - 1]$ and let $d := \gcd(a, m)$ as well as $q = \frac{m}{d}$.
	Next assume that $\left\{x_i\right\}_{i = 1}^q$ is some $q$-term $(\textnormal{mod } m)$-arithmetic progression (not necessarily monochromatic under $\chi$), with gaps which are congruent to $a$ modulo $m$ that is:
	\begin{equation*}
		x_{i + 1} = x_i + d_i \text{ with } d_i \equiv a \mod m
	\end{equation*}
	We know that since $q$ is a divisor of $m$, there exists a unique $q$-element cyclic subgroup $H$ of $\mathbb{Z}_{m}$, by \cite{alg_lauritzen}[Theorem 2.7.4], thus since $\left\{0, d, \ldots, (q - 1)d\right\}$ and $\gen{a}$ are both subgroups, of $\mathbb{Z}_m$, with cardinalities $q$ since $\gcd(a, q) = 1$, by \cite{alg_lauritzen}[Remark 2.7.5], thus we must have:
	\begin{equation*}
		H = \left\{0, d, \ldots, (q - 1)d\right\} = \gen{a}
	\end{equation*}
	It follows that:
	\begin{equation*}
		\left\{[x_1]_{m}, [x_2]_{m}, \ldots, [x_{q}]_{m}\right\} = \left\{[x_1]_{m} + [i a]_{m} | i \in [0; q - 1]\right\} = [x_1]_m + H
	\end{equation*}
	thus $[x_1]_m + H$ a subset of $\mathbb{Z}_{m}$, which is an $q$-term arithmetic progression with gap $d$, thus $\abs{([x_1]_m + H) \cap [0; \ceil{\frac{m}{2}}]} \leq \ceil{\frac{q}{2}}$ and $\abs{([x_1]_m + H) \cap [\ceil{\frac{m}{2}} + 1; m - 1]} \leq \ceil{\frac{q}{2}}$.
	Hence by the definition of $\chi$, no more than $\ceil{\frac{q}{2}}$ members of $[x_1]_m + H$ can be monochromatic.
	However by the definition of $\chi$ we have $\chi(x_i) = \chi([x_{i}]_{m})$ and hence at most $\ceil{\frac{q}{2}}$ elements in $\left\{x_1, x_2, \ldots, x_{q}\right\}$ can be monochromatic.
	The rest follows as $a$ was chosen arbitrarily and $\ceil{\frac{q}{2}} = \ceil{\frac{m}{2\gcd(a, m)}} \leq \ceil{\frac{m}{2}}$.
\end{proof}

