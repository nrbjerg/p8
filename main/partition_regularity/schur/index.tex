\section{Schurs Theorem}
In this section, we will show that for each $k \in \mathbb{N}^{+}$ the configurations
\begin{equation*}
\left\{x_1, x_2, \ldots, x_{k}, \sum_{i = 1}^k x_i\right\} \text{ and }  \left\{x_1, x_2, \ldots, x_{k}, \prod_{i = 1}^k x_i\right\}
\end{equation*}
are partition regular. Our treatment is based on the treatment found in \cite{rtoi}[Chapter 8].
%\begin{enumerate}[label=(S\arabic*), leftmargin=*]
%	\item The configuration $\left\{x_1, x_2, \ldots, x_{k}, \sum_{i = 1}^k x_i\right\}$ is partition regular for every $k \in \mathbb{N}^{+}$. \label{schur_1}
%	\item Let $r, k \in \mathbb{N}^{+}$, then there exists a least positive natural number $S_k(r)$ such that every $r$-coloring of $[1; S_k(r)]$ admits a monochromatic subset of the form $\left\{x_1, x_2, \ldots, x_{k}, \sum_{i = 1}^k x_i\right\}$. \label{schur_2}
%\end{enumerate}
%Statement \ref{schur_2} seems stronger than Statement \ref{schur_1}, since it implies Statement \ref{schur_1}. However they are actually equivalent we need to show the opposite direction, so fix $r, k$ and for each $r$-coloring $\chi$ on $\mathbb{N}$, let
%\begin{equation*}
%	N_{\chi} = \min \left\{\sum_{i = 1}^k x_{i} \middle| \left\{x_1, x_2, \ldots, x_{k}, \sum_{i = 1}^k x_i\right\} \text{ is monochromatic under } \chi \right\}
%\end{equation*}
%(is well defined by \ref{schur_1} and the well ordering of the natural numbers.) The rest should follow by setting $S_k(r) = \max_{\chi} N_{\chi}$. (\textcolor{red}{\textbf{HELP}} I think at least, since although we have an infinite number of $r$-colorings on $\mathbb{N}$ we only have a finite number of $N_{\chi}$'s, since we only have a finite number of colorings if we restrict our selves to $[1; N]$, where $N \in \mathbb{N}^{+}$)

\begin{example}\label{exmp:schur}
	To get a better intuition for the problem, consider the case where $r = 3$ and $k = 2$ and $\mathbb{N}^+$ is colored by:
	\begin{equation*}
		\chi(n) = \begin{cases} red   & \text{ if } n \equiv 0 \mod 3  \\
              blue  & \text{ if }  n \equiv 1 \mod 3 \\
              green & \text{ otherwise }
		\end{cases}
	\end{equation*}
	%A visual representation of $\chi$ would be: $\textcolor{blue}{1}, \textcolor{green}{2}, \textcolor{red}{3}, \textcolor{blue}{4}, \textcolor{green}{5}, \textcolor{red}{6}, \ldots$
	Then $\chi$ admits at least one $red$-monochromatic instance $\left\{3, 3, 6\right\}$ of the configuration $\left\{x, y, x + y\right\}$.
\end{example}

\begin{theorem}[Additive Schur Theorem]\label{thm:additive_schur}
	Let $r, k \in \mathbb{N}^{+}$, then there exists a least natural number $S(k; r)$ such that any $r$-coloring $\chi: [1, S(k; r)] \to C$ admits a monochromatic instance of the configuration $\left\{x_1, x_2, \ldots, x_{k}, \sum_{i = 1}^k x_i\right\}$.
\end{theorem}
\begin{proof}
	We will show that any $r$-coloring $\chi$ of $[1, R(k + 1; r)]$ admits a monochromatic instance of $\left\{x_1, x_2, \ldots, x_{k}, \sum_{i = 1}^k x_i\right\}$. The rest follows directly by the well ordering of $\mathbb{N}^{+}$.

	Let $K$ be the complete graph with vertex set $[1, R(k + 1; r) + 1]$. We will define an $r$-edge coloring $\chi'$ on $K$ by defining $\chi'(\left\{a, b\right\}) := \chi(\abs{a - b})$, by the definition of $R(k + 1; r)$, the $r$-edge coloring $\chi'$ must admit a monochromatic clique $C$ of order $k + 1$. Next we enumerate the verticies $v_1, v_2, \ldots v_{k + 1}$ in $C$ in increasing order
	Then, since $C$ is monochromatic, we see that:
	\begin{equation*}
		\chi(v_i - v_j) = \chi'(\{v_i, v_j\}) = \chi'(\{v_{i'}, v_{j'} \}) = \chi(v_{i'} - v_{j'})
	\end{equation*}
	for all $i > j$ and $i' > j'$, since the verticies are ordered in increasing order. The rest follows by setting $x_j := v_{j + 1} - v_{j}$, notice that $v_{j + 1} - v_j \in [1, R(k + 1; r)]$ and $\sum_{j = 1}^k x_{j} = v_{k + 1} - v_{1} \in [1, R(k + 1; r)]$, since $v_1 < v_{k + 1}$ and $v_{k + 1}, v_1 \in [1, R(k + 1;r) + 1]$.
\end{proof}
Combining the fact that $S(2; r) \leq R(3; r)$ with Corollary \ref{cor:R3r}, we obtain the following upper bound for $S(2; r)$:
\begin{corollary}\label{cor:schur_upper}
  Let $r \in \mathbb{N}^{+}$, then $S(2; r) \leq 3r!$
\end{corollary}
The Schur numbers $S(k; r)$ with $k \neq 2$, is normally refereed to as generalized Schur numbers. The only\footnote{Atleast to the authors knowledge} known values of non-generalized Schur numbers are $S(2; 1) = 2, S(2; 2) = 5, S(2; 3) = 14$, $S(2; 4) = 45$ \cite{fourth_schur_number} and $S(2; 161)$ \cite{fidth_schur_number}. We will primarily be interested in non-generalized Schur numbers, hence we simply refer the reader to \cite{generalized_schur_numbers} for an overview of the known values in the generalized case.
\begin{remark}
	Some authors define the Schur number $S(k; r)$ to be the greatest natural number $N$ such that $[1; N]$ can be colored using $r$-colors without admitting a monochromatic instance of $\left\{x_1, x_2, \ldots, x_{k}, \sum_{i = 1}^k x_i\right\}$. Hence the values presented here and the values found in other literature, may differ by one, depending on the definition of $S(k; r)$.
\end{remark} % S1 = 2, S2 = 5, S3 = 14, S4 = 45, S5 = 161.


%Let $\chi_{0}: \mathbb{N}^+ \to C$ be an $r$-coloring then, by the additive version of Schur's Theorem \ref{thm:additive_schur}, there exists a monochromatic subset $A_0 := \left\{x_1, x_2, \ldots, x_{k}, \sum_{i = 1}^k x_i\right\}$ of $[1, S_{k}(r)]$. Next we may define a new coloring $\chi_1$, which colors every element in $\mathbb{N}^+ \setminus A_{0}$ the same color as $\chi_{0}$, but each element in $A_{0}$ a distinct new color, hence $\chi_1$ is at most a $(r + k + 1)$-coloring, which similarly admits a monochromatic subset $A_1 := \left\{y_1, y_2, \ldots, y_{k}, \sum_{i = 1}^k y_{i}\right\}$ of $[1, S_{k}(r + k + 1)]$. Additionally we note that $\chi_1(n) = \chi_0(n)$ for each $n \in A_1$ since only the elements in $A_0$ are colored a new distinct color, by $\chi_{1}$ compared to $\chi_{0}$. Hence $A_1$ is also monochromatic under $\chi_0$. Repeating this argument we obtain the following corollary:
%\begin{corollary}
%	Let $r, k$ be positive integers, any $r$-coloring $\chi$ of $\mathbb{N}^{+}$ admits infinitely many monochromatic subsets $\mathbb{N}^{+}$ of the form $\left\{x_1, x_2, \ldots, x_{k}, \sum_{i = 1}^k x_i\right\}$.
%\end{corollary}

%\begin{definition}
%	Let $k, r \in \mathbb{N}^{+}$ we define the \textit{Schur number} $S_k(r)$ to be the smallest positive integer such that every $r$-coloring of $[1; S_k(r)]$ admits a monochromatic subset of the form $\left\{x_1, x_2, \ldots, x_{k}, \sum_{i = 1}^{k} x_{i}\right\}$.
%\end{definition}
%Notice that $S_k(r)$ is well defined by Theorem \ref{thm:additive_schur} and the well ordering property of the integers.

The ideas presented in the following definition and in the proof of Theorem \ref{thm:multiplicative_schur} is based upon the ideas found in \cite{exponential_ultrafilters_and_patterns_in_Ramsey_theory}[Section 1].
\begin{definition}
	Let $\chi: \mathbb{N}^{+} \to C$ be a $r$-coloring of $\mathbb{N}^+$, then the \textit{$\log_2$-base coloring} $\chi_{\log_2}: \mathbb{N}^+ \to C$, based on $\chi$, is defined as $\chi_{\log_2}(n) = \chi(2^{n})$.
\end{definition}

%The next theorem is a natural consequence of Theorem \ref{thm:additive_schur}.
\begin{theorem}[Multiplicative Schur Theorem]\label{thm:multiplicative_schur}
	Let $k \in \mathbb{N}^{+}$, then the configuration $\left\{x_1, x_2, \ldots, x_{k}, \prod_{i = 1}^k x_i\right\}$ is partition regular.
\end{theorem}
\begin{proof}
	By the additive Schur Theorem \ref{thm:additive_schur} the $\log_2$-base coloring $\chi_{\log_2}$, based on $\chi$, admits a monochromatic instance of the configuration $\left\{y_1, y_2, \ldots, y_{k}, \sum_{i = 1}^k y_k\right\}$. Hence we must have:
	\begin{equation*}
		\chi(2^{y_1}) = \chi(2^{y_{2}}) = \cdots = \chi(2^{y_k}) = \chi \left(2^{\sum_{i = 1}^k y_{i}}\right)
	\end{equation*}
	the rest follows by setting $x_i = 2^{y_i}$ for $i \in [1; k]$, as $2^{\sum_{i = 1}^k y_k} = \prod_{i = 1}^k 2^{y_i} = \prod_{i = 1}^k x_i$.
\end{proof}
\newpage
%Let $\chi: \mathbb{N}^+ \to C$ be an $r$-coloring on $\mathbb{N}^{+}$.
The additive and multiplicative Schur Theorems \ref{thm:additive_schur} and \ref{thm:multiplicative_schur} respectively, asserts that the configurations $\left\{x_1, x_2, \ldots, x_{k}, \sum^k_{i = 1} x_i\right\}$ and $\left\{x_1, x_2, \ldots, x_{k}, \prod^k_{i = 1} x_i\right\}$ are partition regular. However it is still an open problem, if the configuration $\left\{x_1, x_2, \ldots, x_{k}, \sum_{i = 1}^k x_i, \prod_{i = 1}^k x_i\right\}$ is partition regular, even in the case where $k = 2$. We note that it is relatively easy to check that every $2$-coloring $\chi: [1; 39] \to C$, admits a monochromatic instance of $\left\{x, y, x + y, x \cdot y\right\}$, using the code provided in Appendix \ref{app:schur_code}\footnote{The case where $k = 2$, $r = 3$ was also attempted, however the program crashed due to a lack of RAM. Which the authors suspect is due to the growthrate of the number of possible $3$-colorings on $[1; k]$}.

Next we show that statement of Fermats last theorem: ``The equation $x^n + y^n = z^n$, with $n \geq 2$, has no solution $x, y, z \in \mathbb{N}$, such that $xyz \neq 0$.'' is false if we instead require that $x, y, z$ are non-zero elements in some specific family of finite fields, with sufficiently large characteristics.

%\begin{lemma}\label{lem:ker_of_hom}
%	Let $q$ be a prime, $\phi: \mathbb{F}_q^{*} \to \mathbb{F}_q^*$ be defined by $\phi(x) = x^n$ and $\omega \in \mathbb{F}_q^{*}$ a primitive element, then $\ker(\phi) = \gen{\omega^{\frac{q - 1}{e}}}$ with $e = \gcd(n, q - 1)$.
%\end{lemma}
%\begin{proof}
%	First notice that $\phi(\omega^{\frac{q - 1}{e}}) = \left(\omega^{\frac{q - 1}{e}}\right)^{n} = \left(\omega^{\frac{n}{e}}\right)^{q - 1} = 1$ since $\mathbb{F}_q^{*}$ is cyclic and $\omega$ has order $q - 1$. Hence we have that $\omega^{\frac{q - 1}{e}} \in \ker(\phi)$. Next we note that for all $a \in \ker(\phi)$ there exists an $s \in \mathbb{N}^{+}$ such that $a = \omega^s$, since $\omega$ is a primitive element, and hence $1 = a^n = \omega^{sn}$. Hence $\ord(\omega) = q - 1 | sn$ meaning $e \frac{q - 1}{e} | sn$ which in turn implies that $\frac{q - 1}{e} | s$ since $n$ and $\frac{q - 1}{e}$ are coprime. Hence $a = \omega^{\frac{q -1}{e} \cdot m}$  for some in $m \in \mathbb{N}$, in other words $\ker(\phi) = \gen{\omega^{\frac{q - 1}{e}}}$.
%\end{proof}
%\begin{remark}\label{rem:size_of_ker}
%	In particular Lemma \ref{lem:ker_of_hom} implies that $\abs{\ker(\phi)} = \ord(\omega^{\frac{q - 1}{\gcd(n, q - 1)}}) = \frac{\ord(\omega)}{\gcd \left(\ord(\omega), \frac{q - 1}{\gcd(q - 1, n)}\right)} = \frac{q - 1}{\gcd(q - 1, n)}$.
%\end{remark}
%
%\begin{theorem}
%	Let $n \geq 1$, then there exists a prime $p$ such that for all primes $q \geq p$, the equation $x^n + y^n = z^{n}$ has a solution $x, y, z \in \mathbb{F}_{q}$ with $xyz \neq 0$.
%\end{theorem}
%\begin{proof}
%	Let $q > S_{2}(n)$ be a prime, we will consider the multiplicative group $\mathbb{F}_q^{*}$, and the subgroup $G = \left\{x^n \mid x \in \mathbb{F}_q\right\}$, letting $\phi: \mathbb{F}_q^{*} \ni x \mapsto x^n \in \mathbb{F}_q^{*}$, we see that $\ker(\phi)$ is a subgroup of $G$ more over by the group isomorphism theorem $\mathbb{F}_q^{*} / \ker(\phi) \cong \phi(\mathbb{F}_q^{*}) = G$ and hence:
%	\begin{equation*}
%		\abs{G} = \frac{q - 1}{\frac{q - 1}{\gcd(q - 1, \frac{q-1}{\gcd(q - 1,n)})}} = \frac{q-1}{\gcd(q - 1, n)}
%	\end{equation*}
%	by Langranges Index Theorem and Remark \ref{rem:size_of_ker}. Additionally
%	\begin{equation*}
%		\abs{\mathbb{F}_{q}^{*} / G} = \frac{\abs{\mathbb{F}_{q}}}{\abs{G}} = \frac{q-1}{\frac{q-1}{\gcd(q - 1, n)}} = \gcd(q - 1, n)
%	\end{equation*}
%	Hence there exists $a_1, a_2, \ldots a_k \in \mathbb{F}_q^{*}$ such that $\mathbb{F}_q^{*} = \bigcup_{i = 1}^k a_i G$ with $k := \gcd(n, q - 1)$.
%	Next, we define a $k$-coloring $\chi: \mathbb{F}_q^* \to [k]$ by $\chi(y) = j$ if and only if $y \in a_j G$. Now since $k \leq n$ and $q - 1 \geq S_2(n)$, there exists a monochromatic triple $\left\{x, y, z\right\} \subseteq \mathbb{F}_q^{*}$ such that $x + y = z$, by Theorem \ref{thm:additive_schur}. Meaning there exists an index $j \in [k]$ such that $a_{j}x^{n}, a_jy^{n}, a_{j}z^{n} \in a_jS$ with $a_{j}x^{n} + a_{j}y^{n} = a_jz^{n}$, the rest follows by multiplying by $a_{j}^{-1}$.
%\end{proof}

\begin{theorem}\label{thm:fermat_last_theorem_over_finite_field}
	Let $n \geq 1$, then there exists a prime number $p$ such that for all prime numbers $q \geq p$, the equation $x^n + y^n = z^{n}$ has a solution $x, y, z \in \mathbb{F}_{q}$ with $xyz \neq 0$.
\end{theorem}
\begin{proof}
	Let $q > S(2; n)$ be a prime, we will subgroup $G = \left\{x^n \mid x \in \mathbb{F}_q\right\}$ of the multiplicative group $\mathbb{F}_q^{*}$. Let $\omega$ be a primitive element of $\mathbb{F}_q$, we will show that $G = \gen{\omega^{n}}$, the fact that $\gen{\omega^n} \subseteq G$ follows directly as $\omega^n \in G$. The other inclusion follows since for every $x \in \mathbb{F}_q^{*}$ there exists some $m \in \mathbb{N}$ such that $x = \omega^{m}$ and hence $x^n = \left(\omega^m\right)^n = (\omega^{n})^{m} \in \gen{\omega^{n}}$. Now since $G = \gen{\omega^{n}}$ we see that:
	\begin{equation*}
		\abs{G} = \ord(\omega^{n}) = \frac{\ord(\omega)}{\gcd(\ord(\omega), n)} = \frac{q - 1}{\gcd(q - 1, n)}
	\end{equation*}
	and hence:
	\begin{equation*}
		\abs{\mathbb{F}_q^{*} / G} = \frac{\mathbb{F}^{*}_{q}}{\abs{G}} = \gcd(q - 1, n)
	\end{equation*}
	by Langranges Index Theorem, hence there exists $a_1, a_2, \ldots, a_k \in \mathbb{F}_q^{*}$ such that $\mathbb{F}_q^{*} = \bigcup_{i = 1}^k a_i G$ with $k := \gcd(n, q - 1)$.
	Next, we define a $k$-coloring $\chi: \mathbb{F}_q^* \to [1; k]$ by $\chi(y) = j$ if and only if $y \in a_j G$. Now since $k \leq n$ and $q - 1 \geq S(2; n) \geq S(2; k)$, there exists a monochromatic triple $\left\{x' , y' , z' \right\} \subseteq \mathbb{F}_q^{*}$ such that $x'  + y'  = z' $, by Theorem \ref{thm:additive_schur}, afterall $q - 1 \geq S(2; k)$. Meaning there exists an index $j \in [1; k]$, such that $a_{j}x^{n}, a_jy^{n}, a_{j}z^{n} \in a_jS$, with $a_{j}x^{n} + a_{j}y^{n} = a_jz^{n}$, the rest follows by multiplying by $a_{j}^{-1}$.
\end{proof}

Notably Theorem \ref{thm:fermat_last_theorem_over_finite_field} shows that Fermats Last Theorem cannot be solved, by considering the behavior over finite fields, in the sense that if there where to exists a prime $p$ such that $x^{n} + y^{n} = z^{n}$ has no non-trivial solutions (that is $x, y, z$ such that $xyz \neq 0$) in every finite field $\mathbb{F}_q$ with $q$ a prime greater than $p$, then this would imply Fermats Last Theorem. This can be seen as follows, assume for the sake of contradiction that $x, y, z \in \mathbb{N}$ is a non-trivial solution to $x^n + y^n = z^n$, ensuring that $p \geq \max \left\{x^{n}, y^{n}, z^{n}\right\}$, we see that $x, y, z$ (this time regarded as elements in $\mathbb{F}_q$) gives a solution to $x^n + y^n = z^n$ in every finite field $\mathbb{F}_q$ where $q$ is a prime greater than $p$.
\newpage
\subsection{Lower Bounds on $S(2; r)$ and the Asymptotics of $S(2; r)$ as $r \to \infty$}\label{sub:schur_bounds_and_ass}
In this subsection we will consider the asymptotic behaviour of $S(2; r)$ as $r \to \infty$, although the asymptotic behaviour of $S(2; r)$ is interesting in its own right, the main purpose of our study is to improve on the lower bound of the limit of $\sqrt[k]{R(3; r)}$ as $r \to \infty$, which we established in Section \ref{sec:ramsey_ass}.

Additionally we have already know that $R(k + 1; r) + 1$ is an upper bound of $S(k; r)$, by the proof of Theorem \ref{thm:additive_schur}, we will focus our efforts on establishing lower bounds for Schur numbers, more specifically we will focus on lower bounds for $S(2; r)$, since these are sufficient for our study of the asymptotic behaviour of $S(2; r)$ as $r \to \infty$. Our primary reference will be \cite{emogrt}[Chapter 2].
\begin{definition}
	Let $r, k, N \in \mathbb{N}^+$, with $k \geq 2$ and $\chi: [1; N] \to C$ be an $r$-coloring, then $\chi$ is said to be $k$-sum-free if $\chi$ admits no monochromatic instances of the configuration $\left\{x_1, x_2, \ldots, x_{k}, \sum_{i = 1}^k x_i\right\}$ in $[1; N]$. In the case where $k = 2$, we simply say that $\chi$ is sum-free.
\end{definition}
\begin{remark}\label{rem:sum_free}
	The definition of a sum-free coloring $\chi$, closely resembles the definition of a sum-free set, see Definition \ref{def:set_sum_free}. Also note that if $N < S(k; r)$ we know that there exists at least one such $k$-sum-free $r$-coloring of $[1; N]$, by the definition of the Schur number $S(k; r)$.
\end{remark}
To show that $N \in \mathbb{N}^{+}$ is a lower bound for $S(k; r)$, we look for $r$-colorings of $[1; N]$ which are $k$-sum-free. Our first example of this approach occurs in the following proposition:

\begin{proposition}\label{prop:schur_first_lower_bound}
	Let $r \in \mathbb{N}^{+}$, such that $r \geq 2$, then $S(2; r + 1) > 3S(2; r) - 1$
\end{proposition}
\begin{proof}
	Let $\chi: [1; S(2; r) - 1] \to \left\{c_1, c_2, \ldots, c_{r}\right\}$ be a sum-free $r$-coloring. We will construct a $(r + 1)$-coloring $\gamma: [1; 3S(2; r) - 1] \to \left\{c_1, c_2, \ldots, c_r, c_{r + 1}\right\}$ which we claim is sum-free, we do this by defining $\gamma$ as:
	\begin{equation*}
		\gamma(n) = \begin{cases}
			\chi(n)               & \text{ if }  n \leq S(2; r) - 1          \\
			c_{r + 1}             & \text{ if } n \in [S(2; r); 2S(2; r) - 1] \\
			\chi(3S(2; r) - 1 - n) & \text{ otherwise }
		\end{cases}
	\end{equation*}
	Next we will show that $\gamma$ is indeed sum-free. Hence let $a, b \in [1; 3(S(2; r) - 1)]$ with $\gamma(a) = \gamma(b)$, we have two major cases to consider:
	\begin{enumerate}
		\item If $\gamma(a) = \gamma(b) = c_{r + 1}$ then $a + b \geq 2S(2; r) > 2S(2; r) - 1$ meaning $\gamma(a + b) \neq c_{r + 1}$.
		\item If $\gamma(a) = \gamma(b) = c_{i}$, with $i \neq r + 1$, then we have three subcases:
		      \begin{itemize}
			      \item If $a, b \leq S(2; r) - 1$, we either have $a + b \leq S(2; r) - 1$ or $S(2; r) \leq a + b \leq 2(S(2; r) - 1)$. In the first case we must have $\gamma(a + b) = \chi(a + b) \neq c_{i}$ since $\left\{a, b, a + b\right\}$ would form a $c_i$-monochromatic subset admitted by $\chi$, contradicting the assumption that $\chi$ was sum-free. In the second case we have that $a + b \in [S(2; r); 2S(2; r) - 1]$, and hence $\gamma(a + b) = c_{r + 1}$, so $\left\{a, b,a + b\right\}$ cannot be monochromatic.
			      \item If $a, b \geq 2S(2; r)$ then $a + b \geq 4S(2; r) > 3S(2; r) - 1$, since $S(2; r) \geq 2$ for all $r \in \mathbb{N}^{+}$.
				\item Finally we might have that exactly one of $a$ and $b$ is not larger than that $S(2; r)$ and that the other one is not smaller than $2S(2;r)$. Without loss of generalization we may assume that $a \leq S(2; r) - 1$ and $b \geq 2S(2; r)$. We will let $d := 3S(2; r) - 1 - b$, then $\gamma(b) = \chi(d)$ per the definition of $\gamma$. Next assume for the sake of contradiction that $\gamma(a + b) = c_{i}$, that is $\left\{a, b, a + b\right\}$ forms a $c_i$-monochromatic subset, then since:
			            \begin{equation*}
				            a + b \geq a + 2S(2; r) \geq 2S(2; r) + 1
			            \end{equation*}
			            we have $\gamma(a + b) = \chi(3S(2; r) - 1 - (a + b))$ if we let $d' := 3S(2; r) - 1 - (a + b)$ we see that $a + d' = d$, however since $a, d, d' \in [1; S(2; r) - 1]$ we see that $\{a, d', a + d'\}$ forms a $c_{i}$-monochromatic subset of $[1; S(2; r) - 1]$ under $\chi$ as well, which once again contradicts our assumption that $\chi$ is sum-free. \qedhere
		      \end{itemize}
	\end{enumerate}
\end{proof}

The following theorem is a generalization of Proposition \ref{prop:schur_first_lower_bound}.
\begin{theorem}\label{thm:ass_schur}
	Let $r, r' \in \mathbb{N}^{+}$ then $S(2; r + r') - 1 > (S(2; r) - 1)(S(2; r') - 1) + S(2; r) - 1$.
\end{theorem}
\begin{proof}
	For the sake of convenience we will let $M := 2S(2; r) - 1$ throughout the proof. By Remark \ref{rem:correspondence_between_colorings_and_partition}, it is sufficient to construct a $(r + r')$-partition $D_1, D_2, \ldots, D_{r + r'}$ of $[1; M(S(2; r') - 1) + S(2; r) - 1]$, such that each $D_i$ is sum-free. Let:
	\begin{align*}
		X_c   & := \left\{bM + c | b \in [0; S(2; r') - 1]\right\} \text{ for } c \in [1; S(2; r) - 1] \\
		Y_{b} & := \left\{bM - c | c \in [0; S(2; r) - 1]\right\} \text{ for } b \in [1; S(2; r') - 1]
	\end{align*}
	and
	\begin{equation*}
		\mathcal{X} := \bigcup_{c = 1}^{S(2; r) - 1} X_c, \quad \mathcal{Y} := \bigcup_{b = 1}^{S(2; r') - 1} Y_{b}
	\end{equation*}
	notice that
	\begin{equation*}
		\mathcal{X} = \bigcup_{b = 0}^{S(2; r') - 1} \left\{bM + c | c \in [1; S(2; r) - 1]\right\}
	\end{equation*}
	and hence $[M(S(2; r') - 1) + S(2; r) - 1] = \mathcal{X} \cup \mathcal{Y}$. Finally note that $\mathcal{X}$ and $\mathcal{Y}$ are disjoint.

	We will construct $D_1, D_2, \ldots, D_r$ and $D_{r + 1}, D_{r + 2}, \ldots, D_{r + r'}$ such that they partition $\mathcal{X}$ and $\mathcal{Y}$ respectively. %and does not contain a subset of the form $\left\{x, y, x + y\right\}$. \\
	We will start by constructing $D_1, D_2, \ldots, D_r$. Let $C_1, C_2, \ldots, C_r$ be a sum-free partition of $[1; S(2; r) - 1]$, such a partition exists by Remarks \ref{rem:correspondence_between_colorings_and_partition} and \ref{rem:sum_free}. Let
	\begin{equation*}
		D_i := \bigcup_{c \in C_i} X_c
	\end{equation*}
	for every $i \in [1; S(2; r) - 1]$. Since $X_1, X_2, \ldots, X_{S(2; r) - 1}$ partitions $\mathcal{X}$, we only need to show that every $D_i$ is sum-free, for the sake of contradiction suppose this is not the case, that is there exists an index $i \in [1; S(2; r) - 1]$ and $(b_1M + c_1), (b_2M + c_2), (b_3M + c_3) \in D_{i}$ such that:
	\begin{equation*}
		(b_1 M + c_1) + (b_2M + c_2) = b_3M + c_3
	\end{equation*}
	Then
	\begin{equation*}
		c_1 + c_2 \equiv c_3 \mod M
	\end{equation*}
	however $2 \leq c_1 + c_2 \leq 2(S(2; r) - 1) < M$ and $1 \leq c_3 \leq S(2; r) - 1 < M$, implies that $c_1 + c_2 = c_3$ contradicting the fact that $C_i$ was sum-free.

	Next we will construct $D_{r + 1}, D_{r + 2}, \ldots, D_{r + r'}$. Let $C_{r + 1}, C_{r + 2}, \ldots, C_{r + r'}$ be a sum-free partition of $[1; S(2; r') - 1]$, again such a partition exists by Remarks \ref{rem:correspondence_between_colorings_and_partition} and \ref{rem:sum_free}. We similarly let:
	\begin{equation*}
		D_{r + i} := \bigcup_{b \in C_{r + i}} Y_{b}
	\end{equation*}
	for every $i \in [1; S(2; r') - 1]$. Once again since $Y_0, Y_1, \ldots, Y_{S(2; r') - 1}$ partitions $\mathcal{Y}$, it is sufficient to show that every $D_{r + i}$ is sum-free. Suppose $(b'_1 M - c'_1), (b'_2 M - c'_2), (b'_3M - c'_3) \in D_{r + i}$, then we must have that $b'_1 + b'_2 \neq b'_3$ since $C_{r + i}$ is sum-free. Thus we have two cases:
	\begin{enumerate}
		\item If $b'_1 + b'_2 \geq b'_3 + 1$, then:
		      \begin{equation*}
			      (b'_1 M - c'_1) + (b'_2 M - c'_2) \stackrel{(a)}{\geq} b'_3 M - 1 > b_3 M - c'_3
		      \end{equation*}
		      where $(a)$ follows as $M - c'_1 - c'_2 \leq 2S(2;  r) - 1 - 2(S(2; r) - 1) = 1$.
		\item If $b'_1 + b'_2 \leq b'_3 - 1$, then:
		      \begin{equation*}
			      (b'_1 M - c'_1) + (b'_2 M - c'_2) \stackrel{(b)}{\leq} (b'_3 - 1) M \stackrel{(c)}{<} b_3 M - c'_3
		      \end{equation*}
		      where $(b)$ follows since $c'_1, c'_2 \geq 0$ and $(c)$ as $0 \leq c'_3 < M$.
	\end{enumerate}
	hence every $D_{r + i}$ is sum free, meaning we have constructed a $(r + r')$-partition
	$D_1, D_2, \ldots, D_{r + r'}$ of $[1; M(S(2; r') - 1) + S(2; r) - 1]$ consisting of sum-free sets.
	%we start by defining two ways of partitioning $A := [M(S_2(r') - 1) + S_2(r) - 1]$, for every $b \in [0; S_2(r') - 1]$ let:
	%\begin{equation*}
	%	A_b := \left\{bM + c \middle| c \in [1; S_2(r)]\right\}
	%\end{equation*}
	%and for every $c \in [1; S_2(r) - 1]$ we let:
	%\begin{equation*}
	%	A^{(c)} := \left\{bM + c \middle| b \in [0; S_2(r') - 1]\right\}
	%\end{equation*}
	%Finally let $\chi: [1; M] \to C$ be an sum-free $r$-coloring, we will construct an $r$-coloring $\gamma$ on $A$, by defining $\gamma(a) = \chi(c)$ if and only if $a \in A^{(c)}$. We claim that $\gamma$ is sum-free, suppose this is not the case, then there exists monochromatic elements $b_1M + c_1, b_2M + c_2, b_3M + c_3$, under $\gamma$, such that:
	%\begin{equation*}
	%	(b_1 M + c_1) + (b_2 M + c_2) = (b_3 M + c_3)
	%\end{equation*}
	%where $\chi(c_1) = \chi(c_2) = \chi(c_3)$, thus:
	%\begin{equation*}
	%	c_1 + c_2 \equiv c_3 \mod M
	%\end{equation*}
	%however $2 \leq c_1 + c_2 \leq 2S_{2}(r) - 2 < M$ and $1 \leq c_3 \leq S_{2}(r) - 1 < M$. So $c_1 + c_2 = c_3$ contradicting the fact that $\chi$ is sum-free.
\end{proof}

\begin{corollary}\label{cor:ass_lower_bound}
	The function $\mathbb{N}^{+} \ni r \mapsto 2S(2; r) - 1 \in \mathbb{N}^{+}$ is super-multiplicative and:
	\begin{equation*}
		S(2; r') > c_r(2S(2; r) - 1)^{r' / r}
	\end{equation*}
	for some constant $c_r > 0$, for every $r' \geq r$.
\end{corollary}
\begin{proof}
	The fact that $f$ is super-multiplicative follows directly from Theorem \ref{thm:ass_schur}, since:
	\begin{equation*}
		S(2; r + r') - 1 \geq (2S(2; r) -  1)(S(2; r') - 1) + S(2; r) - 1
	\end{equation*}
	implies:
	\begin{align*}
		2S(2; r + r') - 1 & \geq 2 \left[(2S(2; r) -  1)(S(2; r') - 1) + S(2; r) - 1\right] + 1 \\
		                 & = 4S(2; r)S(2; r') - 2S(2; r') - 2S(2; r) + 1                        \\
		                 & = (2S(2; r) - 1)(2S(2; r') - 1)                                    \\
	\end{align*}
	Furthermore by Lemma \ref{lem:limit_of_super_multiplicative} there exists a constant $c_r > 0$ such that:
	\begin{equation*}
		2S(2; r') - 1 \geq c'_r (2S(2; r) - 1)^{r' / r}
	\end{equation*}
	and hence by setting $c_r := \frac{c'_{r}}{2}$, we see that:
	\begin{equation*}
		S(2; r') > S(2; r') - \frac{1}{2} \geq c_{r} (2S(2; r) - 1)^{r' / r} \qedhere
	\end{equation*}
\end{proof}

Proposition \ref{prop:schur_first_lower_bound} gives that $S(2; 6) > 3S(2; 5) - 1 = 482$, since $S(2; 5) = 161$. However this lower bound has been improved, in \cite{lower_bound_for_s6}, to $S(2; 6) \geq 537$. Hence by Corollary \ref{cor:ass_lower_bound}, we have $S(2; r) \geq c_6 1073^{r / 6}$, for some constant $c_6 > 0$ and all $r \geq 6$. In particular since $S(2; r) \leq R(3; r)$, by the proof of Theorem \ref{thm:additive_schur}, we have $R(3; r) \geq c_6 1073^{r / 6} > c_6 3.199^{r}$, for all $r \geq 6$, since $1073^{1 / 6} > 3.199$.

%\subsection{Bounds on Schur Numbers}
\begin{remark}\label{rem:initial_bound_on_schur}
	By the proof of Theorem \ref{thm:additive_schur}, we have $S_k(r) \leq R(k + 1; r)$.
\end{remark}
\begin{definition}
	Let $r, k, N \in \mathbb{N}^+$, with $k \geq 2$ and $\chi: [1; N] \to C$ be an $r$-coloring, then $\chi$ is said to be $k$-sum-free if there exists no monochromatic subset of $[1; N]$ of the form $\left\{x_1, x_2, \ldots, x_{k}, \sum_{i = 1}^k x_i\right\}$. In the case where $k = 2$, we simply say that $\chi$ is sum-free.
\end{definition}
\begin{remark}\label{rem:sum_free}
	If $N < S_k(r)$ we know that there exists at least one such $k$-sum-free $r$-coloring of $[1; N]$, by the definition of $S_k(r)$.
\end{remark}
To show that $N \in \mathbb{N}$ is a lower bound for $S_{k}(r)$, we look for $r$-colorings of $[1; N]$ which is not $k$-sum-free. Our first example of this approach occurs in the following theorem, the proof of this is based on the proof found in \cite{emogrt}[Chapter 2].
\begin{theorem}[]
	Let $r \in \mathbb{N}^{+}$, such that $r \geq 2$, then $S_{2}(r + 1) \geq 3(S_2(r) - 1)$
\end{theorem}
\begin{proof}
	Let $\chi: [S_{2}(r) - 1] \to \left\{c_1, c_2, \ldots, c_{r}\right\}$ be an $r$-coloring which emits no monochromatic subsets of the form $\left\{x, y, x + y\right\}$. We will construct a $(r + 1)$-coloring $\gamma: [1; 3(S_2(r) - 1)] \to \left\{c_1, c_2, \ldots, c_r, c_{r + 1}\right\}$ which similarly emits no monochromatic subsets of the form $\left\{x, y, x + y\right\}$, we do this by setting:
	\begin{equation*}
		\gamma(n) = \begin{cases}
			\chi(n)               & \text{ if }  n \leq S_2(r) - 1          \\
			c_{r + 1}             & \text{ if } n \in [S_2(r); 2S_2(r) - 1] \\
			\chi(3S_2(r) - 2 - n) & \text{ otherwise }
		\end{cases}
	\end{equation*}
	We will show that $\gamma$ emits no $c_{r + 1}$-monochromatic subset of the form $\left\{x, y, x + y\right\}$, hence let $x, y \in [1; 3(S_2(r) - 1)]$ with $\gamma(x) = \gamma(y)$, two major cases to consider:
	\begin{enumerate}
		\item $\gamma(x) = \gamma(y) = c_{r + 1}$ implies $x + y \geq 2S_2(r) > 2S_2(r) - 1$ meaning $\gamma(x + y) \neq c_{r + 1}$.
		\item $\gamma(x) = \gamma(y) = c_{i}$, with $i \neq r + 1$, in which case we have three subcases:
		      \begin{itemize}
			      \item If $x, y \leq S_2(r) - 1$, we either have $x + y \leq S_2(r) - 1$ or $x + y \geq 2S_2(r)$. In the first case we must have $\gamma(x + y) = \chi(x + y) \neq c_{i}$ since $\left\{x, y, x + y\right\}$ would form a $c_i$-monochromatic subset emitted by $\chi$. In the second case we see that $x + y \leq 2(S_2(r) - 1) < 2S_2(r) - 1$, so $\gamma(x + y) = c_{r + 1}$, so $\left\{x, y, x + y\right\}$ cannot be monochromatic.
			      \item If $x, y \geq 2S_2(r)$ then $x + y \geq 2S_2(r) + 2S_2(r) = 4S_2(r) > 3S(r) - 3$.
			      \item Finally if precisely one of $x$ and $y$ is less than that $S_2(r)$ and one is greater than $2S_2(r)$, without loss of generalization we may assume that $x \leq S_2(r) - 1$ and $y \geq 2S_2(r)$. We will let $z := 3S_2(r) - 2 - y$, then $\gamma(y) = \chi(z)$ per the definition of $\gamma$. Next assume for the sake of contradiction that $\gamma(x + y) = c_{i}$, that is $\left\{x, y, x + y\right\}$ forms a $c_i$-monochromatic subset, then since:
			            \begin{equation*}
				            x + y \geq x + 2S_2(r) \geq 2S_2(r) + 1
			            \end{equation*}
			            we have $\gamma(x + y) = \chi(3S_2(r) - 2 - (x + y))$ if we let $z' := 3S_2(r) - 2 - (x + y)$ we see that $x + z' = z$, however since $x, z, z' \in [1; S_2(r) - 1]$ we see that $\{x, z', x + z'\}$ forms a $c_{i}$-monochromatic subset of $[1; S_2(r) - 1]$ under $\chi$ aswell. \qedhere
		      \end{itemize}
	\end{enumerate}
\end{proof}

%\subsection{Assymptotic behaviour $S_{2}(k)$ and its relation to $R(3;k)$}\label{sec:schur_ass}

